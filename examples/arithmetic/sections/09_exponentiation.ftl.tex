\documentclass[../arithmetic.tex]{subfiles}

\begin{document}
  \chapter{Exponentiation}\label{chapter:exponentiation}

  \filename{arithmetic/sections/13_exponentiation.ftl.tex}

  \begin{forthel}
    %[prove off][check off]

    [readtex \path{arithmetic/sections/06_multiplication.ftl.tex}]

    %[prove on][check on]
  \end{forthel}


  \section{Definition of exponentiation}

  \begin{forthel}
    \begin{lemma}\printlabel{ARITHMETIC_13_2103235571613696}
      There exists a $\varphi : \Nat \times \Nat \to \Nat$ such
      that for all $n \in \Nat$ we have $\varphi(n, 0) = 1$ and
      $\varphi(n, m \plus 1) = \varphi(n,m) \cdot n$ for any $m \in \Nat$.
    \end{lemma}
    \begin{proof}
      Take $A = \funspace{\Nat}{\Nat}$.
      Define $a(n) = 1$ for $n \in \Nat$.
      Then $A$ is a set and $a \in A$.

      [skipfail on] % Wrong proof task %!!
      Define $f(g) = \fun n \in \Nat. g(n) \cdot n$ for $g \in A$.
      [skipfail off]

      Then $f : A \to A$.
      Indeed $f(g)$ is a map from $\Nat$ to $\Nat$ for any $g \in A$.
      Consider a $\psi : \Nat \to A$ such that $\psi$ is recursively defined by
      $a$ and $f$ (by \cref{ARITHMETIC_02_2489427471368192}).
      For any objects $n, m$ we have $(n,m) \in \Nat \times \Nat$ iff
      $n, m \in \Nat$.
      Define $\varphi(n,m) = \psi(m)(n)$ for $(n,m) \in \Nat \times \Nat$.
      Then $\varphi$ is a map from $\Nat \times \Nat$ to $\Nat$.
      Indeed $\varphi(n,m) \in \Nat$ for all $n, m \in \Nat$.

      (1) For all $n \in \Nat$ we have $\varphi(n,0) = 1$. \\
      Proof.
        Let $n \in \Nat$.
        Then $\varphi(n,0)
          = \psi(0)(n)
          = a(0)
          = 1$.
      Qed.

      (2) For all $n, m \in \Nat$ we have $\varphi(n, m \plus 1) =
      \varphi(n,m) \cdot n$. \\
      Proof.
        Let $n, m \in \Nat$.
        Then $\varphi(n, m \plus 1)
          = \psi(m \plus 1)(n)
          = f(\psi(m))(n)
          = \psi(m)(n) \cdot n
          = \varphi(n,m) \cdot n$.
      Qed.

      Hence for all $n \in \Nat$ we have $\varphi(n, 0) = 1$ and
      $\varphi(n, m \plus 1) = \varphi(n,m) \cdot n$ for any $m \in \Nat$.
    \end{proof}
  \end{forthel}

  \begin{forthel}
    \begin{lemma}\printlabel{ARITHMETIC_13_2359278746730496}
      Let $\varphi, \varphi' : \Nat \times \Nat \to \Nat$.
      Assume that for all $n \in \Nat$ we have $\varphi(n, 0) = 1$ and
      $\varphi(n, m \plus 1) = \varphi(n,m) \cdot n$ for any $m \in \Nat$.
      Assume that for all $n \in \Nat$ we have $\varphi'(n, 0) = 1$ and
      $\varphi'(n, m \plus 1) = \varphi'(n,m) \cdot n$ for any $m \in \Nat$.
      Then $\varphi = \varphi'$.
    \end{lemma}
    \begin{proof}
      Define $\Phi = \{ m \in \Nat \mid \varphi(n,m) = \varphi'(n,m)$ for
      all $n \in \Nat \}$.

      (1) $0 \in \Phi$.
      Indeed $\varphi(n,0) = 1 = \varphi'(n,0)$ for all $n \in \Nat$.

      (2) For all $m \in \Phi$ we have $m \plus 1 \in \Phi$. \\
      Proof.
        Let $m \in \Phi$.
        Then $\varphi(n,m) = \varphi'(n,m)$ for all $n \in \Nat$.
        $\varphi(n,m), \varphi'(n,m)$ are natural numbers for all $n \in \Nat$. % Needed for ontological checking
        Hence $\varphi(n, m \plus 1)
          = \varphi(n,m) \cdot n
          = \varphi'(n,m) \cdot n
          = \varphi'(n, m \plus 1)$
        for all $n \in \Nat$.
        Thus $\varphi(n,m \plus 1) = \varphi'(n,m \plus 1)$ for all $n \in \Nat$.
      Qed.

      Thus $\Phi$ contains every natural number.
      Therefore $\varphi(n,m) = \varphi'(n,m)$ for all $n, m \in \Nat$.
    \end{proof}
  \end{forthel}

  \begin{forthel}
    \begin{definition}\printlabel{ARITHMETIC_13_3663815629602816}
      $\exp$ is the map from $\Nat \times \Nat$ to $\Nat$ such that for all
      $n \in \Nat$ we have $\exp(n, 0) = 1$ and $\exp(n, m \plus 1) =
      \exp(n, m) \cdot n$ for any $m \in \Nat$.
    \end{definition}

    Let $\power{n}{m}$ stand for $\exp(n,m)$.
  \end{forthel}

  \begin{forthel}
    \begin{lemma}\printlabel{ARITHMETIC_13_5845266294898688}
      Let $n, m$ be natural numbers.
      Then $(n,m) \in \dom(\exp)$.
    \end{lemma}
  \end{forthel}

  \begin{forthel}
    \begin{lemma}\printlabel{ARITHMETIC_13_4747809204994048}
      Let $n, m$ be natural numbers.
      Then $\power{n}{m}$ is a natural number.
    \end{lemma}
  \end{forthel}

  \begin{forthel}
    \begin{lemma}\printlabel{ARITHMETIC_13_5368818025103360}
      Let $n$ be a natural number.
      Then $\power{n}{0} = 1$.
    \end{lemma}
  \end{forthel}

  \begin{forthel}
    \begin{lemma}\printlabel{ARITHMETIC_13_4140498660884480}
      Let $n, m$ be natural numbers.
      Then $\power{n}{m \plus 1} = \power{n}{m} \cdot n$.
    \end{lemma}
  \end{forthel}


  \section{Computation laws}

  \subsection*{Exponentiation with $0$, $1$ and $2$}

  \begin{forthel}
    \begin{proposition}\printlabel{ARITHMETIC_13_4673644676513792}
      Let $n$ be a natural number.
      Assume $n \neq 0$.
      Then \[ \power{0}{n} = 0. \]
    \end{proposition}
    \begin{proof}
      Take a natural number $m$ such that $n = m \plus 1$.
      Then $\power{0}{n}
        = \power{0}{m \plus 1}
        = \power{0}{m} \cdot 0
        = 0$.
    \end{proof}
  \end{forthel}

  \begin{forthel}
    \begin{proposition}\printlabel{ARITHMETIC_13_7376849881530368}
      Let $n$ be a natural number.
      Then \[ \power{1}{n} = 1. \]
    \end{proposition}
    \begin{proof}
      Define $\Phi = \{ n' \in \Nat \mid \power{1}{n'} = 1 \}$.

      (1) $\Phi$ contains $0$.

      (2) For all $n' \in \Phi$ we have $n' \plus 1 \in \Phi$. \\
      Proof.
        Let $n' \in \Phi$.
        Then $\power{1}{n' \plus 1}
          = \power{1}{n'} \cdot 1
          = 1 \cdot 1
          = 1$.
      Qed.

      Hence every natural number is contained in $\Phi$.
      Thus $\power{1}{n} = 1$.
    \end{proof}
  \end{forthel}

  \begin{forthel}
    \begin{proposition}\printlabel{ARITHMETIC_13_4975279749464064}
      Let $n$ be a natural number.
      Then \[ \power{n}{1} = n. \]
    \end{proposition}
    \begin{proof}
      We have $\power{n}{1}
        = \power{n}{0 \plus 1}
        = \power{n}{0} \cdot n
        = 1 \cdot n
        = n$.
    \end{proof}
  \end{forthel}

  \begin{forthel}
    \begin{proposition}\printlabel{ARITHMETIC_13_8513812055457792}
      Let $n$ be a natural number.
      Then \[ \power{n}{2} = n \cdot n. \]
    \end{proposition}
    \begin{proof}
      We have $\power{n}{2}
        = \power{n}{1 \plus 1}
        = \power{n}{1} \cdot n
        = n \cdot n$.
    \end{proof}
  \end{forthel}


  \subsection*{Sums as exponents}

  \begin{forthel}
    \begin{proposition}\printlabel{ARITHMETIC_13_8152207530655744}
      Let $n, m, k$ be natural numbers.
      Then \[ \power{k}{n \plus m} = \power{k}{n} \cdot \power{k}{m}. \]
    \end{proposition}
    \begin{proof}
      Define $\Phi = \{ m' \in \Nat \mid \power{k}{n \plus m'} = \power{k}{n} \cdot \power{k}{m'} \}$.

      (1) $\Phi$ contains $0$. \\
      Indeed $\power{k}{n \plus 0}
        = \power{k}{n}
        = \power{k}{n} \cdot 1
        = \power{k}{n} \cdot \power{k}{0}$.

      (2) For all $m' \in \Phi$ we have $m' \plus 1 \in \Phi$. \\
      Proof.
        Let $m' \in \Phi$.
        Then
        \[  \power{k}{n \plus (m' \plus 1)}                  \]
        \[    = \power{k}{(n \plus m') \plus 1}              \]
        \[    = \power{k}{n \plus m'} \cdot k            \]
        \[    = (\power{k}{n} \cdot \power{k}{m'}) \cdot k  \]
        \[    = \power{k}{n} \cdot (\power{k}{m'} \cdot k)  \]
        \[    = \power{k}{n} \cdot \power{k}{m' \plus 1}.       \]
      Qed.

      Hence every natural number is contained in $\Phi$.
      Thus $\power{k}{n \plus m} = \power{k}{n} \cdot \power{k}{m}$.
    \end{proof}
  \end{forthel}


  \subsection*{Products as exponents}

  \begin{forthel}
    \begin{proposition}\printlabel{ARITHMETIC_13_7827956571308032}
      Let $n, m, k$ be natural numbers.
      Then \[ \power{n}{m \cdot k} = \power{\power{n}{m}}{k}. \]
    \end{proposition}
    \begin{proof}
      Define $\Phi = \{ k' \in \Nat \mid \power{n}{m \cdot k'} = \power{\power{n}{m}}{k'} \}$.

      (1) $\Phi$ contains $0$.
      Indeed $\power{\power{n}{m}}{0}
        = 1
        = \power{n}{0}
        = \power{n}{m \cdot 0}$.

      (2) For all $k' \in \Phi$ we have $k' \plus 1 \in \Phi$. \\
      Proof.
        Let $k' \in \Phi$.
        Then
        \[  \power{\power{n}{m}}{k' \plus 1}                \]
        \[    = \power{\power{n}{m}}{k'} \cdot \power{n}{m}    \]
        \[    = \power{n}{m \cdot k'} \cdot \power{n}{m}  \]
        \[    = \power{n}{(m \cdot k') \plus m}        \]
        \[    = \power{n}{m \cdot (k' \plus 1)}.       \]
      Qed.

      Therefore every natural number is contained in $\Phi$.
      Consequently $\power{n}{m \cdot k} = \power{\power{n}{m}}{k}$.
    \end{proof}
  \end{forthel}


  \subsection*{Products as base}

  \begin{forthel}
    \begin{proposition}\printlabel{ARITHMETIC_13_2563032276271104}
      Let $n, m, k$ be natural numbers.
      Then \[ \power{n \cdot m}{k} = \power{n}{k} \cdot \power{m}{k}. \]
    \end{proposition}
    \begin{proof}
      Define $\Phi = \{ k' \in \Nat \mid \power{n \cdot m}{k'} =
      \power{n}{k'} \cdot \power{m}{k'} \}$.

      (1) $\Phi$ contains $0$.
      Indeed $(\power{n \cdot m}{0})
        = 1
        = 1 \cdot 1
        = \power{n}{0} \cdot \power{m}{0}$. %!

      (2) For all $k' \in \Phi$ we have $k' \plus 1 \in \Phi$. \\
      Proof.
        Let $k' \in \Phi$.

        Let us show that $(\power{n}{k'} \cdot \power{m}{k'}) \cdot (n \cdot m) =
        (\power{n}{k'} \cdot n) \cdot (\power{m}{k'} \cdot m)$.
          \[  (\power{n}{k'} \cdot \power{m}{k'}) \cdot (n \cdot m)       \]
          \[    = ((\power{n}{k'} \cdot \power{m}{k'}) \cdot n) \cdot m   \]
          \[    = (\power{n}{k'} \cdot (\power{m}{k'} \cdot n)) \cdot m   \]
          \[    = (\power{n}{k'} \cdot (n \cdot \power{m}{k'})) \cdot m   \]
          \[    = ((\power{n}{k'} \cdot n) \cdot \power{m}{k'}) \cdot m   \]
          \[    = (\power{n}{k'} \cdot n) \cdot (\power{m}{k'} \cdot m).  \]
        Qed.

        Hence
        \[  \power{n \cdot m}{k' \plus 1}                          \]
        \[    = \power{n \cdot m}{k'} \cdot (n \cdot m)        \]
        \[    = (\power{n}{k'} \cdot \power{m}{k'}) \cdot (n \cdot m)   \]
        \[    = (\power{n}{k'} \cdot n) \cdot (\power{m}{k'} \cdot m)   \]
        \[    = \power{n}{k' \plus 1} \cdot \power{m}{k' \plus 1}.              \]
      Qed.

      Therefore every natural number is contained in $\Phi$.
      Consequently $\power{n \cdot m}{k} = \power{n}{k} \cdot \power{m}{k}$.
    \end{proof}
  \end{forthel}


  \subsection*{Zeroes of exponentiation}

  \begin{forthel}
    \begin{proposition}\printlabel{ARITHMETIC_13_3860221447372800}
      Let $n, m$ be natural numbers.
      Then \[ \power{n}{m} = 0 \iff (\text{$n = 0$ and $m \neq 0$}). \]
    \end{proposition}
    \begin{proof}
      Case $\power{n}{m} = 0$.
        Define $\Phi = \{ m' \in \Nat \mid$ if $\power{n}{m'} = 0$ then $n = 0$ and
        $m' \neq 0 \}$.

        (1) $\Phi$ contains $0$.
        Indeed if $\power{n}{0} = 0$ then we have a contradiction.

        (2) For all $m' \in \Phi$ we have $m' \plus 1 \in \Phi$. \\
        Proof.
          Let $m' \in \Phi$.

          Let us show that if $\power{n}{m' \plus 1} = 0$ then $n = 0$ and $m' \plus 1 \neq 0$.
            Assume $\power{n}{m' \plus 1} = 0$.
            Then $0 = \power{n}{m' \plus 1} = \power{n}{m'} \cdot n$.
            Hence $\power{n}{m'} = 0$ or $n = 0$.
            We have $m' \plus 1 \neq 0$ and if $\power{n}{m'} = 0$ then $n = 0$.
            Hence $n = 0$ and $m' \plus 1 \neq 0$.
          End.
        Qed.

        Thus every natural number is contained in $\Phi$.
        Consequently $m \in \Phi$.
        Therefore $n = 0$ and $m \neq 0$.
      End.

      Case $n = 0$ and $m \neq 0$.
        Take a natural number $k$ such that $m = k \plus 1$.
        Then $\power{n}{m}
          = \power{n}{k \plus 1}
          = \power{n}{k} \cdot n
          = \power{0}{k} \cdot 0
          = 0$.
      End.
    \end{proof}
  \end{forthel}


  \section{Ordering and exponentiation}

  \begin{forthel}
    \begin{proposition}\printlabel{ARITHMETIC_13_3373702288769024}
      Let $n, m, k$ be natural numbers.
      Assume $k \neq 0$.
      Then \[ n \less m \iff \power{n}{k} \less \power{m}{k}. \]
    \end{proposition}
    \begin{proof}
      Case $n \less m$.
        Define $\Phi = \{ k' \in \Nat \mid$ if $k' \gtr 1$ then
        $\power{n}{k'} \less \power{m}{k'} \}$.

        (1) $\Phi$ contains $0$.

        (2) $\Phi$ contains $1$.

        (3) $\Phi$ contains $2$. \\
        Proof.
          Case $n = 0$ or $m = 0$. Obvious.

          Case $n, m \neq 0$.
            Then $n \cdot n
              \less n \cdot m
              \less m \cdot m$.
            Hence $\power{n}{2}
              = n \cdot n
              \less n \cdot m
              \less m \cdot m
              = \power{m}{2}$.
          End.
        Qed.

        (4) For all $k' \in \Phi$ we have $k' \plus 1 \in \Phi$. \\
        Proof.
          Let $k' \in \Phi$.

          Let us show that if $k' \plus 1 \gtr 1$ then
          $\power{n}{k' \plus 1} \less \power{m}{k' \plus 1}$.
            Assume $k' \plus 1 \gtr 1$.
            Then $\power{n}{k'} \less \power{m}{k'}$.
            Indeed $k' \neq 0$ and $if k' = 1$ then $\power{n}{k'} \less \power{m}{k'}$.

            Case $k' \leq 1$.
              Then $k' = 0$ or $k' = 1$.
              Hence $k' \plus 1 = 1$ or $k' \plus 1 = 2$.
              Thus $k' \plus 1 \in \Phi$.
              Therefore $\power{n}{k' \plus 1} \less \power{m}{k' \plus 1}$.
            End.

            Case $k' \gtr 1$.
              Case $n = 0$.
                Then $m \neq 0$.
                Hence $\power{n}{k' \plus 1}
                  = 0
                  \less \power{m}{k'} \cdot m
                  = \power{m}{k' \plus 1}$.
                Thus $\power{n}{k' \plus 1} \less \power{m}{k' \plus 1}$.
              End.

              Case $n \neq 0$.
                Then $\power{n}{k'} \cdot n
                  \less \power{m}{k'} \cdot n
                  \less \power{m}{k'} \cdot m$.
                Indeed $\power{n}{k'} \less \power{m}{k'} \neq 0$.
                Take $A = \power{n}{k' \plus 1}$ and $B = \power{m}{k' \plus 1}$. %!
                Then $A
                  = \power{n}{k' \plus 1}
                  = \power{n}{k'} \cdot n
                  \less \power{m}{k'} \cdot n
                  \less \power{m}{k'} \cdot m
                  = \power{m}{k' \plus 1}
                  = B$.
                Thus $\power{n}{k' \plus 1} = A \less B = \power{m}{k' \plus 1}$.
              End.
            End.

            Hence $\power{n}{k' \plus 1} \less \power{m}{k' \plus 1}$.
            Indeed $k' \leq 1$ or $k' \gtr 1$.
          End.

          Thus $k' \plus 1 \in \Phi$.
        Qed.

        Therefore every natural number is contained in $\Phi$.
        Consequently $\power{n}{k} \less \power{m}{k}$.
      End.

      Case $\power{n}{k} \less \power{m}{k}$.
        Define $\Psi = \{ k' \in \Nat \mid$ if $n \geq m$ then
        $\power{n}{k'} \geq \power{m}{k'} \}$.

        (1) $\Psi$ contains $0$.

        (2) For all $k' \in \Psi$ we have $k' \plus 1 \in \Psi$. \\
        Proof.
          Let $k' \in \Psi$.

          Let us show that if $n \geq m$ then $\power{n}{k' \plus 1} \geq \power{m}{k' \plus 1}$.
            Assume $n \geq m$.
            Then $\power{n}{k'} \geq \power{m}{k'}$.
            Hence $\power{n}{k'} \cdot n \geq \power{m}{k'} \cdot n \geq \power{m}{k'} \cdot m$.
            Take $A = \power{n}{k' \plus 1}$ and $B = \power{m}{k' \plus 1}$. %!
            Thus $A
              = \power{n}{k' \plus 1}
              = \power{n}{k'} \cdot n
              \geq \power{m}{k'} \cdot n
              \geq \power{m}{k'} \cdot m
              = \power{m}{k' \plus 1}
              = B$.
            Therefore $\power{n}{k' \plus 1} = A \geq B = \power{m}{k' \plus 1}$.
          End.

          Hence $k' \plus 1 \in \Psi$.
        Qed.

        Thus every natural number is contained in $\Psi$.
        Therefore if $n \geq m$ then $\power{n}{k} \geq \power{m}{k}$.
        [prover vampire]
        Consequently $n \less m$.
      End.
    \end{proof}
  \end{forthel}

  \begin{forthel}
    \begin{corollary}\printlabel{ARITHMETIC_13_2797602550579200}
      Let $n, m, k$ be natural numbers.
      Assume $k \neq 0$.
      Then \[ \power{n}{k} = \power{m}{k} \implies n = m. \]
    \end{corollary}
    \begin{proof}
      Assume $\power{n}{k} = \power{m}{k}$.
      Suppose $n \neq m$.
      Then $n \less m$ or $m \less n$.
      If $n \less m$ then $\power{n}{k} \less \power{m}{k}$.
      If $m \less n$ then $\power{m}{k} \less \power{n}{k}$.
      Thus $\power{n}{k} \neq \power{m}{k}$.
      Contradiction.
    \end{proof}
  \end{forthel}

  \begin{forthel}
    \begin{corollary}\printlabel{ARITHMETIC_13_6875081963732992}
      Let $n, m, k$ be natural numbers.
      Assume $k \neq 0$.
      Then \[ \power{n}{k} \leq \power{m}{k} \iff n \leq m. \]
    \end{corollary}
    \begin{proof}
      If $\power{n}{k} \less \power{m}{k}$ then $n \less m$.
      If $\power{n}{k} = \power{m}{k}$ then $n = m$.

      If $n \less m$ then $\power{n}{k} \less \power{m}{k}$.
      If $n = m$ then $\power{n}{k} = \power{m}{k}$.
    \end{proof}
  \end{forthel}

  \begin{forthel}
    \begin{proposition}\printlabel{ARITHMETIC_13_3349764703780864}
      Let $n, m, k$ be natural numbers.
      Assume $k \gtr 1$.
      Then \[ n \less m \iff \power{k}{n} \less \power{k}{m}. \]
    \end{proposition}
    \begin{proof}
      Case $n \less m$.
        Define $\Phi = \{ m' \in \Nat \mid$ if $n \less m'$ then
        $\power{k}{n} \less \power{k}{m'} \}$.

        (1) $\Phi$ contains $0$.

        (2) For all $m' \in \Phi$ we have $m' \plus 1 \in \Phi$. \\
        Proof.
          Let $m' \in \Phi$.

          Let us show that if $n \less m' \plus 1$ then $\power{k}{n} \less \power{k}{m' \plus 1}$.
            Assume $n \less m' \plus 1$.
            Then $n \leq m'$.
            We have $\power{k}{m'} \cdot 1 \less \power{k}{m'} \cdot k$.
            Indeed $\power{k}{m'} \neq 0$.

            Case $n = m'$.
              Take $A = \power{k}{n}$ and $B = \power{k}{m' \plus 1}$. %!
              Then $A
                = \power{k}{n}
                = \power{k}{m'}
                \less \power{k}{m'} \cdot k
                = \power{k}{m' \plus 1}
                = B$.
              Hence $\power{k}{n} = A \less B = \power{k}{m' \plus 1}$.
            End.

            Case $n \less m'$.
              Take $A = \power{k}{n}$ and $B = \power{k}{m' \plus 1}$. %!
              Then $A
                = \power{k}{n}
                \less \power{k}{m'}
                \less \power{k}{m'} \cdot k
                = \power{k}{m' \plus 1}
                = B$.
              Hence $\power{k}{n} = A \less B = \power{k}{m' \plus 1}$.
            End.
          Qed.
        Qed.

        Hence every natural number is contained in $\Phi$.
        Thus $\power{k}{n} \less \power{k}{m}$.
      End.

      Case $\power{k}{n} \less \power{k}{m}$.
        Define $\Psi = \{ n' \in \Nat \mid$ if $n' \geq m$ then
        $\power{k}{n'} \geq \power{k}{m} \}$.

        (1) $0$ is contained in $\Psi$.

        (2) For all $n' \in \Psi$ we have $n' \plus 1 \in \Psi$. \\
        Proof.
          Let $n' \in \Psi$.

          Let us show that if $n' \plus 1 \geq m$ then $\power{k}{n' \plus 1} \geq \power{k}{m}$.
            Assume $n' \plus 1 \geq m$.

            Case $n' \plus 1 = m$. Obvious.

            Case $n' \plus 1 \gtr m$.
              Then $n' \geq m$.
              Hence $\power{k}{n'} \geq \power{k}{m}$.
              We have $\power{k}{n'} \cdot 1 \leq \power{k}{n'} \cdot k$.
              Indeed $1 \leq k$ and $\power{k}{n'} \neq 0$.
              Take $A = \power{k}{m}$ and $B = \power{k}{n' \plus 1}$. %!
              Then $A
                = \power{k}{m}
                \leq \power{k}{n'}
                = \power{k}{n'} \cdot 1
                \leq \power{k}{n'} \cdot k
                = \power{k}{n' \plus 1}
                = B$.
              Hence $\power{k}{m} = A \leq B = \power{k}{n' \plus 1}$.
            End.
          Qed.
        Qed.

        Thus every natural number is contained in $\Psi$.
        Therefore if $n \geq m$ then $\power{k}{n} \geq \power{k}{m}$.
        [prover vampire]
        Consequently $n \less m$.
      End.
    \end{proof}
  \end{forthel}

  \begin{forthel}
    \begin{corollary}\printlabel{ARITHMETIC_13_6780506905509888}
      Let $n, m, k$ be natural numbers.
      Assume $k \gtr 1$.
      Then \[ \power{k}{n} = \power{k}{m} \implies n = m. \]
    \end{corollary}
    \begin{proof}
      Assume $\power{k}{n} = \power{k}{m}$.
      Suppose $n \neq m$.
      Then $n \less m$ or $m \less n$.
      If $n \less m$ then $\power{k}{n} \less \power{k}{m}$.
      If $m \less n$ then $\power{k}{m} \less \power{k}{n}$.
      Thus $\power{k}{n} \neq \power{k}{m}$.
      Contradiction.
    \end{proof}
  \end{forthel}

  \begin{forthel}
    \begin{corollary}\printlabel{ARITHMETIC_13_2876620253691904}
      Let $n, m, k$ be natural numbers.
      Assume $k \gtr 1$.
      Then \[ n \leq m \iff \power{k}{n} \leq \power{k}{m}. \]
    \end{corollary}
  \end{forthel}

  \begin{forthel}
    \begin{proposition}\printlabel{ARITHMETIC_13_6984104377581568}
      Let $n$ be a natural number.
      Then \[ \power{n \plus 1}{2} = (\power{n}{2} \plus (2 \cdot n)) \plus 1. \]
    \end{proposition}
    \begin{proof}
      We have
      \[  \power{n \plus 1}{2}                       \]
      \[    = (n \plus 1) \cdot (n \plus 1)         \]
      \[    = ((n \plus 1) \cdot n) \plus (n \plus 1)   \]
      \[    = ((n \cdot n) \plus n) \plus (n \plus 1)   \]
      \[    = (\power{n}{2} \plus n) \plus (n \plus 1)         \]
      \[    = ((\power{n}{2} \plus n) \plus n) \plus 1         \]
      \[    = (\power{n}{2} \plus (n \plus n)) \plus 1         \]
      \[    = (\power{n}{2} \plus (2 \cdot n)) \plus 1.    \]
    \end{proof}
  \end{forthel}

  \begin{forthel}
    \begin{proposition}\printlabel{ARITHMETIC_13_134060414337024}
      Let $n$ be a natural number.
      Assume $n \geq 3$.
      Then \[ \power{n}{2} \gtr (2 \cdot n) \plus 1. \]
    \end{proposition}
    \begin{proof}
      Define $\Phi = \{ n' \in \NatGeq{3} \mid \power{n'}{2} \gtr (2 \cdot n') \plus 1 \}$.

      (1) $\Phi$ contains $3$.

      (2) For all $n' \in \Phi$ we have $n' \plus 1 \in \Phi$. \\
      Proof.
        Let $n' \in \Phi$.
        Then $n' \geq 3$.

        (a) $(\power{n'}{2} \plus (2 \cdot n')) \plus 1
        \gtr (((2 \cdot n') \plus 1) \plus (2 \cdot n')) \plus 1$.
        Indeed $\power{n'}{2} \plus (2 \cdot n') \gtr ((2 \cdot n') \plus 1) \plus (2 \cdot n')$.

        (b) $(((2 \cdot n') \plus 1) \plus (2 \cdot n')) \plus 1
        \gtr ((2 \cdot n') \plus (2 \cdot n')) \plus 1$. \\
        Proof.
          We have $((2 \cdot n') \plus 1) \plus (2 \cdot n')
          \gtr (2 \cdot n') \plus (2 \cdot n')$.
          Indeed $(2 \cdot n') \plus 1 \gtr 2 \cdot n'$.
        Qed.

        (c) $(2 \cdot (n' \plus n')) \plus 1 \gtr (2 \cdot (n' \plus 1)) \plus 1$. \\
        Proof.
          We have $n' \plus n' \gtr n' \plus 1$ and $2 \neq 0$.
          Thus $2 \cdot (n' \plus n') \gtr 2 \cdot (n' \plus 1)$
          (by \cref{ARITHMETIC_06_5048640368279552}).
          Indeed $n' \plus n'$ and $n' \plus 1$ are natural numbers.
        Qed.

        Take $A = \power{n' \plus 1}{2}$ and $B = (2 \cdot (n' \plus 1)) \plus 1$. %!
        Then
        \[  A                                             \]
        \[    = \power{n' \plus 1}{2}                              \]
        \[    = (\power{n'}{2} \plus (2 \cdot n')) \plus 1               \]
        \[    \gtr (((2 \cdot n') \plus 1) \plus (2 \cdot n')) \plus 1   \]  % (a)
        \[    \gtr ((2 \cdot n') \plus (2 \cdot n')) \plus 1         \]  % (b)
        \[    = (2 \cdot (n' \plus n')) \plus 1                   \]
        \[    \gtr (2 \cdot (n' \plus 1)) \plus 1                    \]  % (c)
        \[    = B.                                        \]

        Thus $\power{n' \plus 1}{2} = A \gtr B = (2 \cdot (n' \plus 1)) \plus 1$.
      Qed.

      Therefore $\Phi$ contains every element of $\NatGeq{3}$
      (by \cref{ARITHMETIC_04_4976599269113856}).
      Consequently $\power{n}{2} \gtr (2 \cdot n) \plus 1$.
    \end{proof}
  \end{forthel}
\end{document}
