\documentclass[../arithmetic.tex]{subfiles}

\begin{document}
  \chapter{Ordering}\label{chapter:ordering}

  \filename{arithmetic/sections/04_ordering.ftl.tex}

  \begin{forthel}
    %[prove off][check off]

    [readtex \path{foundations/sections/11_binary-relations.ftl.tex}]

    [readtex \path{arithmetic/sections/03_addition.ftl.tex}]

    %[prove on][check on]
  \end{forthel}


  \section{Definitions and immediate consequences}

  \begin{forthel}
    \begin{definition}\printlabel{ARITHMETIC_04_1926295512416256}
      Let $n, m$ be natural numbers.
      $n \less m$ iff there exists a nonzero natural number $k$ such that
      $m = n \plus k$.
    \end{definition}

    Let $n$ is less than $m$ stand for $n \less m$.
    Let $n \gtr m$ stand for $m \less n$.
    Let $n$ is greater than $m$ stand for $n \gtr m$.
    Let $n \nless m$ stand for $n$ is not less than $m$.
    Let $n \ngtr m$ stand for $n$ is not greater than $m$.
  \end{forthel}

  \begin{forthel}
    \begin{definition}\printlabel{ARITHMETIC_04_3668680374222848}
      Let $n$ be a natural number.
      $\NatLess{n} = \{ k \in \Nat \mid k \less n \}$.
    \end{definition}
  \end{forthel}

  \begin{forthel}
    \begin{definition}\printlabel{ARITHMETIC_04_3670333934534656}
      Let $n$ be a natural number.
      $\NatGtr{n} = \{ k \in \Nat \mid k \gtr n \}$.
    \end{definition}
  \end{forthel}

  \begin{forthel}
    \begin{definition}\printlabel{ARITHMETIC_04_7916616566177792}
      Let $n$ be a natural number.
      $n$ is positive iff $n \gtr 0$.
    \end{definition}
  \end{forthel}

  \begin{forthel}
    \begin{definition}\printlabel{ARITHMETIC_04_4593841531256832}
      Let $n, m$ be natural numbers.
      $n \leq m$ iff there exists a natural number $k$ such that $m = n \plus k$.
    \end{definition}

    Let $n$ is less than or equal to $m$ stand for $n \leq m$.
    Let $n \geq m$ stand for $m \leq n$.
    Let $n$ is greater than or equal to $m$ stand for $n \geq m$.
    Let $n \nleq m$ stand for $n$ is not less than or equal to $m$.
    Let $n \ngeq m$ stand for $n$ is not greater than or equal to $m$.
  \end{forthel}

  \begin{forthel}
    \begin{definition}\printlabel{ARITHMETIC_04_72501526790144}
      Let $n$ be a natural number.
      $\NatLeq{n} = \{ k \in \Nat \mid k \leq n \}$.
    \end{definition}
  \end{forthel}

  \begin{forthel}
    \begin{definition}\printlabel{ARITHMETIC_04_1706933421604864}
      Let $n$ be a natural number.
      $\NatGeq{n} = \{ k \in \Nat \mid k \geq n \}$.
    \end{definition}
  \end{forthel}

  \begin{forthel}
    \begin{proposition}\printlabel{ARITHMETIC_04_5385415374667776}
      Let $n, m$ be natural numbers.
      $n \leq m$ iff $n \less m$ or $n = m$.
    \end{proposition}
    \begin{proof}
      Case $n \leq m$.
        Take a natural number $k$ such that $m = n \plus k$.
        If $k = 0$ then $n = m$. If $k \neq 0$ then $n \less m$.
      End.

      Case $n \less m$ or $n = m$.
        If $n \less m$ then there is a positive natural number $k$ such that
        $m = n \plus k$.
        If $n = m$ then $m = n \plus 0$.
        Thus if $n \less m$ then there is a natural number $k$ such that
        $m = n \plus k$.
      End.
    \end{proof}
  \end{forthel}

  \begin{forthel}
    \begin{definition}\printlabel{ARITHMETIC_04_6232154608500736}
      Let $n$ be a natural number.
      A predecessor of $n$ is a natural number that is less than $n$.
    \end{definition}
  \end{forthel}

  \begin{forthel}
    \begin{definition}\printlabel{ARITHMETIC_04_8147686326796288}
      Let $n$ be a natural number.
      A successor of $n$ is a natural number that is greater than $n$.
    \end{definition}
  \end{forthel}

  \begin{forthel}
    \begin{proposition}\printlabel{ARITHMETIC_04_4826285599621120}
      Let $n$ be a natural number.
      Then $n$ is positive iff $n$ is nonzero.
    \end{proposition}
    \begin{proof}
      Case $n$ is positive.
        Take a positive natural number $k$ such that $n = 0 \plus k = k$.
        Then we have $n \neq 0$.
      End.

      Case $n$ is nonzero.
        Take a natural number $k$ such that $n = k \plus 1$.
        Then $n = 0 \plus (k \plus 1)$.
        $k \plus 1$ is positive.
        Hence $0 \less n$.
      End.
    \end{proof}
  \end{forthel}


  \section{Basic properties}

  \begin{forthel}
    \begin{proposition}\printlabel{ARITHMETIC_04_1037693395927040}
      Let $n$ be a natural number.
      Then \[ n \nless n. \]
    \end{proposition}
    \begin{proof}
      Assume the contrary.
      Then we can take a positive natural number $k$ such that $n = n \plus k$.
      Then we have $0 = k$.
      Contradiction.
    \end{proof}
  \end{forthel}

  \begin{forthel}
    \begin{proposition}\printlabel{ARITHMETIC_04_8266284905005056}
      Let $n, m$ be natural numbers.
      Then \[ n \less m \implies n \neq m. \]
    \end{proposition}
    \begin{proof}
      Assume $n \less m$.
      Take a positive natural number $k$ such that $m = n \plus k$.
      If $n = m$ then $k = 0$.
      Hence $n \neq m$.
    \end{proof}
  \end{forthel}

  \begin{forthel}
    \begin{proposition}\printlabel{ARITHMETIC_04_4190604718243840}
      Let $n, m$ be natural numbers.
      Then \[ (\text{$n \leq m$ and $m \leq n$}) \implies n = m. \]
    \end{proposition}
    \begin{proof}
      Assume $n \leq m$ and $m \leq n$.
      Take natural numbers $k, l$ such that $m = n \plus k$ and $n = m \plus l$.
      Then $m
        = (m \plus l) \plus k
        = m \plus (l \plus k)$.
      Hence $l \plus k = 0$.
      Thus $l = 0 = k$.
      Indeed if $l \neq 0$ or $k \neq 0$ then $l \plus k$ is the direct successor of
      some natural number.
      Therefore $m = n$.
    \end{proof}
  \end{forthel}

  \begin{forthel}
    \begin{proposition}\printlabel{ARITHMETIC_04_6413905244979200}
      Let $n, m, k$ be natural numbers.
      Then \[ n \less m \less k \implies n \less k. \]
    \end{proposition}
    \begin{proof}
      Assume $n \less m \less k$.
      Take a positive natural number $a$ such that $m = n \plus a$.
      Take a positive natural number $b$ such that $k = m \plus b$.
      Then $k
        = (n \plus a) \plus b
        = n \plus (a \plus b)$.
      $a \plus b$ is positive.
      Hence $n \less k$.
    \end{proof}
  \end{forthel}

  \begin{forthel}
    \begin{proposition}\printlabel{ARITHMETIC_04_5480385953660928}
      Let $n, m, k$ be natural numbers.
      Then \[ n \leq m \leq k \implies n \leq k. \]
    \end{proposition}
    \begin{proof}
      Assume $n \leq m \leq k$.
      Case $n = m = k$. Obvious.
      Case $n = m \less k$. Obvious.
      Case $n \less m = k$. Obvious.
      Case $n \less m \less k$. Obvious.
    \end{proof}
  \end{forthel}

  \begin{forthel}
    \begin{proposition}\printlabel{ARITHMETIC_04_5098403656630272}
      Let $n, m, k$ be natural numbers.
      Then \[ n \leq m \less k \implies n \less k. \]
    \end{proposition}
    \begin{proof}
      Assume $n \leq m \less k$.
      If $n = m$ then $n \less k$.
      If $n \less m$ then $n \less k$.
    \end{proof}
  \end{forthel}

  \begin{forthel}
    \begin{proposition}\printlabel{ARITHMETIC_04_4809599527944192}
      Let $n, m, k$ be natural numbers.
      Then \[ n \less m \leq k \implies n \less k. \]
    \end{proposition}
    \begin{proof}
      Assume $n \less m \leq k$.
      If $m = k$ then $n \less k$.
      If $m \less k$ then $n \less k$.
    \end{proof}
  \end{forthel}

  \begin{forthel}
    \begin{proposition}\printlabel{ARITHMETIC_04_8584998051381248}
      Let $n, m$ be natural numbers.
      Then \[ n \less m \implies n \plus 1 \leq m. \]
    \end{proposition}
    \begin{proof}
      Assume $n \less m$.
      Take a positive natural number $k$ such that $m = n \plus k$.

      Case $k = 1$.
        Then $m = n \plus 1$.
        Hence $n \plus 1 \leq m$.
      End.

      Case $k \neq 1$.
        Then we can take a natural number $l$ such that $k = l \plus 1$.
        Then $m
          = n \plus (l \plus 1)
          = (n \plus l) \plus 1
          = (n \plus 1) \plus l$.
        $l$ is positive.
        Thus $n \plus 1 \less m$.
      End.
    \end{proof}
  \end{forthel}

  \begin{forthel}
    \begin{proposition}\printlabel{ARITHMETIC_04_8201937860165632}
      Let $n, m$ be natural numbers.
      Then $n \less m$ or $n = m$ or $n \gtr m$.
    \end{proposition}
    \begin{proof}
      Define $\Phi = \{ m' \in \Nat \mid n \less m'$ or $n = m'$ or $n \gtr m' \}$.

      (1) $\Phi$ contains $0$.

      (2) For all $m' \in \Phi$ we have $m' \plus 1 \in \Phi$. \\
      Proof.
        Let $m' \in \Phi$.

        Case $n \less m'$. Obvious.

        Case $n = m'$. Obvious.

        Case $n \gtr m'$.
          Take a positive natural number $k$ such that $n = m' \plus k$.

          Case $k = 1$. Obvious.

          Case$k \neq 1$.
            Take a natural number $l$ such that $n = (m' \plus 1) \plus l$.
            Hence $n \gtr m' \plus 1$.
            Indeed $l$ is positive.
          End.
        Qed.
      Qed.

      Thus every natural number is contained in $\Phi$.
      Therefore $n \less m$ or $n = m$ or $n \gtr m$.
    \end{proof}
  \end{forthel}

  \begin{forthel}
    \begin{proposition}\printlabel{ARITHMETIC_04_6991525988794368}
      Let $n, m$ be natural numbers.
      Then \[ n \nless m \iff n \geq m. \]
    \end{proposition}
    \begin{proof}
      Case $n \nless m$.
        Then $n = m$ or $n \gtr m$.
        Hence $n \geq m$.
      End.

      Case $n \geq m$.
        Assume $n \less m$.
        Then $n \leq m$.
        Hence $n = m$.
        Contradiction.
      End.
    \end{proof}
  \end{forthel}


  \section{Ordering and successors}

  \begin{forthel}
    \begin{proposition}\printlabel{ARITHMETIC_04_7006203091615744}
      Let $n, m$ be natural numbers.
      Then \[ n \less m \leq n \plus 1 \implies m = n \plus 1. \]
    \end{proposition}
    \begin{proof}
      Assume $n \less m \leq n \plus 1$.
      Take a positive natural number $k$ such that $m = n \plus k$.
      Take a natural number $l$ such that $n \plus 1 = m \plus l$.
      Then $n \plus 1
        = m \plus l
        = (n \plus k) \plus l
        = n \plus (k \plus l)$.
      Hence $k \plus l = 1$.

      We have $l = 0$. \\
      Proof.
        Assume the contrary.
        Then $k,l \gtr 0$.

        Case $k,l = 1$.
          Then $k \plus l
            = 2
            \neq 1$.
          Contradiction.
        End.

        Case $k = 1 and l \neq 1$.
          Then $l \gtr 1$.
          Hence $k \plus l
            \gtr 1 \plus l
            \gtr 1$.
          Contradiction.
        End.

        Case $k \neq 1 and l = 1$.
          Then $k \gtr 1$.
          Hence $k \plus l
            \gtr k \plus 1
            \gtr 1$.
          Contradiction.
        End.

        Case $k, l \neq 1$.
          Take natural numbers $a, b$ such that $k = a \plus 1$ and $l = b \plus 1$.
          Indeed $k, l \neq 0$.
          Hence $k = a \plus 1$ and $l = b \plus 1$.
          Thus $k, l \gtr 1$. Indeed $a, b$ are positive.
        End.
      Qed.

      Then we have $n \plus 1
        = m \plus l
        = m \plus 0
        = m$.
    \end{proof}
  \end{forthel}

  \begin{forthel}
    \begin{proposition}\printlabel{ARITHMETIC_04_8792330561650688}
      Let $n, m$ be natural numbers.
      Then \[ n \leq m \less n \plus 1 \implies n = m. \]
    \end{proposition}
    \begin{proof}
      Assume $n \leq m \less n \plus 1$.

      Case $n = m$. Obvious.

      Case $n \less m$.
        Then $n \less m \leq n \plus 1$.
        Hence $n = m$.
      End.
    \end{proof}
  \end{forthel}

  \begin{forthel}
    \begin{corollary}\printlabel{ARITHMETIC_04_1802826644717568}
      Let $n$ be a natural number.
      There is no natural number $m$ such that $n \less m \less n \plus 1$.
    \end{corollary}
    \begin{proof}
      Assume the contrary.
      Take a natural number $m$ such that $n \less m \less n \plus 1$.
      Then $n \less m \leq n \plus 1$ and $n \leq m \less n \plus 1$.
      Hence $m = n \plus 1$ and $m = n$.
      Hence $n = n \plus 1$.
      Contradiction.
    \end{proof}
  \end{forthel}

  \begin{forthel}
    \begin{proposition}\printlabel{ARITHMETIC_04_990407185924096}
      Let $n$ be a natural number.
      Then \[ n \plus 1 \geq 1. \]
    \end{proposition}
    \begin{proof}
      Case $n = 0$. Obvious.

      Case $n \neq 0$.
        Then $n \gtr 0$.
        Hence $n \plus 1 \gtr 0 \plus 1 = 1$.
      End.
    \end{proof}
  \end{forthel}


  \section{Ordering and addition}

  \begin{forthel}
    \begin{proposition}\printlabel{ARITHMETIC_04_7354062662008832}
      Let $n, m, k$ be natural numbers.
      Then \[ n \less m \iff n \plus k \less m \plus k. \]
    \end{proposition}
    \begin{proof}
      Case $n \less m$.
        Take a positive natural number $l$ such that $m = n \plus l$.
        Then $m \plus k
          = (n \plus l) \plus k
          = (n \plus k) \plus l$.
        Hence $n \plus k \less m \plus k$.
      End.

      Case $n \plus k \less m \plus k$.
        Take a positive natural number $l$ such that $m \plus k = (n \plus k) \plus l$.
        $(n \plus k) \plus l
          = n \plus (k \plus l)
          = n \plus (l \plus k)
          = (n \plus l) \plus k$.
        Hence $m \plus k = (n \plus l) \plus k$.
        Thus $m = n \plus l$.
        Therefore $n \less m$.
      End.
    \end{proof}
  \end{forthel}

  \begin{forthel}
    \begin{corollary}\printlabel{ARITHMETIC_04_1901366129721344}
      Let $n, m, k$ be natural numbers.
      Then \[ n \less m \iff k \plus n \less k \plus m. \]
    \end{corollary}
    \begin{proof}
      We have $k \plus n = n \plus k$ and $k \plus m = m \plus k$.
      Hence $k \plus n \less k \plus m$ iff $n \plus k \less m \plus k$.
    \end{proof}
  \end{forthel}

  \begin{forthel}
    \begin{corollary}\printlabel{ARITHMETIC_04_4203390999461888}
      Let $n, m, k$ be natural numbers.
      Then \[ n \leq m \iff k \plus n \leq k \plus m. \]
    \end{corollary}
  \end{forthel}

  \begin{forthel}
    \begin{corollary}\printlabel{ARITHMETIC_04_5512590832697344}
      Let $n, m, k$ be natural numbers.
      Then \[ n \leq m \iff n \plus k \leq m \plus k. \]
    \end{corollary}
  \end{forthel}


  \section{The natural numbers are well-ordered}

  \begin{forthel}
    \begin{definition}\printlabel{ARITHMETIC_04_4059354166722560}
      \[ \LessRel = \{ (n, m) \mid \text{$n$ and $m$ are natural numbers such that
      $n \less m$} \}. \]
    \end{definition}
  \end{forthel}

  \begin{forthel}
    \begin{proposition}\printlabel{ARITHMETIC_04_5933477660721152}
      Let $A$ be a nonempty subclass of $\Nat$.
      Let $n, m$ be least elements of $A$ regarding $\LessRel$.
      Then $n = m$.
    \end{proposition}
    \begin{proof}
      Assume $n \neq m$.
      Then $n \less m$ or $m \less n$.
      If $n \less m$ then $n \notin A$.
      If $m \less n$ then $m \notin A$.
      Hence $n, m \notin A$.
      Contradiction.
      Therefore $n = m$.
    \end{proof}
  \end{forthel}

  \begin{forthel}
    \begin{proposition}\printlabel{ARITHMETIC_04_272317502455808}
      Let $A$ be a nonempty subclass of $\Nat$.
      Then $A$ has a least element regarding $\LessRel$.
    \end{proposition}
    \begin{proof}
      Assume the contrary.

      Let us show that for each $n \in A$ there exists a $m \in A$ such that
      $m \less n$.
        Let $n \in A$.
        $A$ has no least element regarding $\LessRel$.
        Assume that there exists no $m \in A$ such that $m \less n$.
        Then $n \leq m$ for all $m \in A$.
        Hence $n$ is a least element of $A$ regarding $\LessRel$.
        Contradiction.
      End.

      Define $\Phi = \{ n \in \Nat \mid n$ is less than any element of $A \}$.

      (1) $\Phi$ contains $0$. \\
      Proof.
        $0 \notin A$.
        Hence $0$ is less than every element of $A$.
        Thus $0 \in \Phi$.
      Qed.

      (2) For all $n \in \Phi$ we have $n \plus 1 \in \Phi$. \\
      Proof.
        Let $n \in \Phi$.
        Then $n$ is less than any element of $A$.
        Assume that $\Phi$ does not contain $n \plus 1$.
        Then we can take an $m \in A$ such that $n \plus 1 \nless m$.
        Then $n \less m \leq n \plus 1$.
        Hence $m = n \plus 1$.
        Thus $n \plus 1$ is a least element of $A$ regarding $\LessRel$.
        Contradiction.
      Qed.

      Then $\Phi$ contains every natural number.
      Therefore every natural number is less than any element of $A$.
      Consequently $A$ is empty.
      Contradiction.
    \end{proof}
  \end{forthel}

  \begin{forthel}
    \begin{corollary}\printlabel{ARITHMETIC_04_4280275783647232}
      $\LessRel$ is a wellorder on every nonempty subclass of $\Nat$.
    \end{corollary}
    \begin{proof}
      Let $A$ be a nonempty subclass of $\Nat$.
      For any $n, m \in A$ we have $(n, m) \in \LessRel$ iff $n \less m$.

      (1) $\LessRel$ is irreflexive on $A$.
      Indeed for any $n \in A$ we have $n \nless n$.

      (2) $\LessRel$ is transitive on $A$.
      Indeed for any $n, m, k \in A$ if $n \less m$ and $m \less k$ then $n \less k$.

      (3) $\LessRel$ is connected on $A$.
      Indeed for any distinct $n, m \in A$ we have $n \less m$ or $m \less n$.

      Hence $\LessRel$ is a strict linear order on $A$.
      $\LessRel$ is wellfounded on $A$.
      Indeed every nonempty subclass of $A$ has a least element regarding $\LessRel$.
      Thus $\LessRel$ is a wellorder on $A$.
    \end{proof}
  \end{forthel}


  \section{Induction revisited}

  \begin{forthel}
    \begin{theorem}\printlabel{ARITHMETIC_04_3609801697263616}
      Let $A$ be a class.
      Assume for all $n \in \Nat$ if $A$ contains all predecessors of $n$ then
      $A$ contains $n$.
      Then $A$ contains every natural number.
    \end{theorem}
    \begin{proof}
      Assume the contrary.
      Take a natural number $n$ that is not contained in $A$.
      Then $n$ is contained in $\Nat \setminus A$.
      Hence we can take a least element $m$ of $\Nat \setminus A$ regarding
      $\LessRel$.
      Then $\Nat \setminus A$ does not contain any predecessor of $m$.
      Therefore $A$ contains all predecessors of $m$.
      Consequently $A$ contains $m$.
      Contradiction.
    \end{proof}
  \end{forthel}

  \begin{forthel}
    \begin{theorem}\printlabel{ARITHMETIC_04_4976599269113856}
      Let $A$ be a class.
      Let $k$ be a natural number such that $k \in A$.
      Assume that for all $n \in \NatGeq{k}$ if $n \in A$ then $n \plus 1 \in A$.
      Then for all $n \in \NatGeq{k}$ we have $n \in A$.
    \end{theorem}
    \begin{proof}
      Define $\Phi = \{n \in \Nat \mid$ if $n \geq k$ then $n \in A \}$.

      (1) $\Phi$ contains $0$.
      Indeed if $0 \geq k$ then $0 = k \in A$.

      (2) For all $n \in \Phi$ we have $n \plus 1 \in \Phi$. \\
      Proof.
        Let $n \in \Phi$.

        Let us show that if $n \plus 1 \geq k$ then $n \plus 1 \in A$.
          Assume $n \plus 1 \geq k$.

          Case $n \less k$.
            Then $n \plus 1 = k$.
            Hence $n \plus 1 \in A$.
          End.

          Case $n \geq k$.
            Then $n \in A$.
            Hence $n \plus 1 \in A$.
          End.
        End.

        Therefore $n \plus 1 \in \Phi$.
      Qed.

      Thus $\Phi$ contains every natural number.
      Consequently for all $n \in \NatGeq{k}$ we have $n \in A$.
    \end{proof}
  \end{forthel}
\end{document}
