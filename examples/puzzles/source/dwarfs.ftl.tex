\documentclass{article}
\usepackage[lang=en]{stex}
\libusepackage{naproche-logo}
\libinput{preamble}

\title{Dwarfs and Hats}
\author{\Naproche Formalization: Steffen Frerix and Peter Koepke}
\date{2018 and 2021}

\begin{document}

\maketitle

\begin{smodule}{dwarfs.ftl}
  \section{Introduction}

  We present a formalized solution in \Naproche of a simple “hat problem”:

  \begin{quotation}
    \noindent The two dwarfs Sigbert and Tormund are on an expedition and get captured by an indigenous tribe.
    In order to be released they are given the following challenge:
    Both of them will be placed in a room and will be given a hat.
    They can only see the hat the other dwarf is wearing.
    The hats are either white or black.
    Without further communication, one of the dwarfs must be able to name the color of his hat.
    The two may agree on a strategy before they are placed in the room.
    How can they manage to be released?
  \end{quotation}


  \section{Dwarf Ontology}

  \importmodule[naproche/examples/preliminaries]{everyday-ontology.ftl}

  \begin{forthel}
    \begin{signature}
      Sigbert is a dwarf.
    \end{signature}
    \begin{signature}
      Tormund is a dwarf.
    \end{signature}
    \begin{axiom}
      The color of the hat of any dwarf is White or Black.
    \end{axiom}
    \begin{axiom}
      The opposite color of White is Black.
    \end{axiom}
    \begin{axiom}
      The opposite color of Black is White.
    \end{axiom}
  \end{forthel}


  \section{The Challenge}

  \begin{forthel}
    \begin{axiom}
      If some dwarf $D$ names the color of the hat of $D$ then all dwarfs get released.
    \end{axiom}
  \end{forthel}


  \section{The Strategy of the Dwarfs}

  \begin{forthel}
    \begin{axiom}
      Sigbert names the opposite color of the color of the hat of Tormund.
    \end{axiom}
    \begin{axiom}
      Tormund names the color of the hat of Sigbert.
    \end{axiom}
  \end{forthel}


  \section{Release!}

  \begin{forthel}
    \begin{theorem}
      All dwarfs get released.
    \end{theorem}
  \end{forthel}
\end{smodule}

\end{document}
