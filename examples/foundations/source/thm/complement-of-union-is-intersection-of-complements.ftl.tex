\documentclass{article}
\usepackage[stex]{naproche}
\libinput{preamble}
\begin{document}
\begin{smodule}[creators={Marcel Schütz}]{complement-of-union-is-intersection-of-complements.ftl}
  \importmodule[naproche/examples/foundations]{def/classes?intersection.ftl}
  \importmodule[naproche/examples/foundations]{def/classes?union.ftl}
  \importmodule[naproche/examples/foundations]{def/classes?complement.ftl}
  \importmodule[naproche/examples/foundations]{thm?subclass-antisym.ftl}

  \begin{fproposition*}[label=718948240719872]
    Let $A, B, C$ be classes.
    Then $\complement{A}{\union{B,C}} = \interesction{\complement{A}{B},\complement{A}{C}}$.
  \end{fproposition*}
  \begin{fproof}
    Let us show that $\complement{A}{\union{B,C}} \subclass[eq] \intersection{\complement{A}{B},\complement{A}{C}}$.
      Let $x \in \complement{A}{\union{B,C}}$.
      Then $x \in A$ and $x \notin \union{B,C}$.
      Hence it is wrong that ($x \in B$ or $x \in C$).
      Thus $x \notin B$ and $x \notin C$.
      Therefore $x \in A$ and ($x \notin B$ and $x \notin C$).
      Then ($x \in A$ and $x \notin B$) and ($x \in A$ and $x \notin C$).
      Hence $x \in \complement{A}{B}$ and $x \in \complement{A}{C}$.
      Thus $x \in \intersection{\complement{A}{B},\complement{A}{C}}$.
    End.

    Let us show that $\intersection{\complement{A}{B},\complement{A}{C}} \subclass[eq] \complement{A}{\union{B,C}}$.
      Let $x \in \intersection{\complement{A}{B},\complement{A}{C}}$.
      Then $x \in \complement{A}{B}$ and $x \in \complement{A}{C}$.
      Hence ($x \in A$ and $x \notin B$) and ($x \in A$ and $x \notin C$).
      Thus $x \in A$ and ($x \notin B$ and $x \notin C$).
      Therefore $x \in A$ and not ($x \in B$ or $x \in C$).
      Then $x \in A$ and not $x \in \union{B,C}$.
      Hence $x \in \complement{A}{\union{B,C}}$.
    End.
  \end{fproof}
\end{smodule}
\end{document}
