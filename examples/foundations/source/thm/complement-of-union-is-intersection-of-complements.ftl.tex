\documentclass{article}
\usepackage[stex]{naproche}
\libinput{preamble}
\begin{document}
\begin{smodule}[creators={Marcel Schütz}]{complement-of-union-is-intersection-of-complements.ftl}
  \importmodule[naproche/examples/foundations]{def/classes?intersection.ftl}
  \importmodule[naproche/examples/foundations]{def/classes?union.ftl}
  \importmodule[naproche/examples/foundations]{def/classes?complement.ftl}
  \importmodule[naproche/examples/foundations]{thm?subclass-antisym.ftl}

  \begin{fproposition*}[label=718948240719872]
    Let $A, B, C$ be classes.
    Then $A \complement (B \union C) = (A \complement B) \intersection (A \complement C)$.
  \end{fproposition*}
  \begin{fproof}
    Let us show that $A \complement (B \union C) \subclass[eq]
    (A \complement B) \intersection (A \complement C)$.
      Let $x \in A \complement (B \union C)$.
      Then $x \in A$ and $x \notin B \union C$.
      Hence it is wrong that ($x \in B$ or $x \in C$).
      Thus $x \notin B$ and $x \notin C$.
      Therefore $x \in A$ and ($x \notin B$ and $x \notin C$).
      Then ($x \in A$ and $x \notin B$) and ($x \in A$ and $x \notin C$).
      Hence $x \in A \complement B$ and $x \in A \complement C$.
      Thus $x \in (A \complement B) \intersection (A \complement C)$.
    End.

    Let us show that $((A \complement B) \intersection (A \complement C)) \subclass[eq]
    A \complement (B \union C)$. %!
      Let $x \in (A \complement B) \intersection (A \complement C)$.
      Then $x \in A \complement B$ and $x \in A \complement C$.
      Hence ($x \in A$ and $x \notin B$) and ($x \in A$ and $x \notin C$).
      Thus $x \in A$ and ($x \notin B$ and $x \notin C$).
      Therefore $x \in A$ and not ($x \in B$ or $x \in C$).
      Then $x \in A$ and not $x \in B \union C$.
      Hence $x \in A \complement (B \union C)$.
    End.
  \end{fproof}
\end{smodule}
\end{document}
