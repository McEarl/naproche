\documentclass[12pt,oneside]{book}

\usepackage[utf8]{inputenc}
\usepackage[english]{babel}

\usepackage{../meta-inf/lib/naproche-logo}
\usepackage{../meta-inf/lib/naproche}
\usepackage{../meta-inf/lib/libraries}
\usepackage{biblatex}
\addmhbibresource{references}

\libusepackage{naproche}

\newcommand{\pow}{\mathcal{P}}

\usepackage{graphicx}
\usepackage{float}
\usepackage{caption}
\usepackage[nobottomtitles]{titlesec}

\setlength{\headheight}{15pt}


\title{Foundations of Mathematics}
\author{Marcel Schütz}
\date{2022}

\begin{document}
  \maketitle

  \tableofcontents

  \begin{figure}[H]
    \centering
    \fbox{\includegraphics[width=0.9\linewidth]{./dependency-graph/graph.png}}
    \caption*{Interdependencies of the chapters}
  \end{figure}


  \section*{Introduction}

  This is a library providing a foundation of mathematics based on a
  Kelley-Morse like class theory with urelements.
  It introduces common operations on classes like unions or intersections
  (\cref{chapter:classes}) together with detailed proofs of their algebraic
  properties (\cref{chapter:computation-laws-for-classes}), the symmetric
  difference of two classes (\cref{chapter:symmetric-difference}) and the
  notions of ordered pairs and Cartesian products
  (\cref{chapter:pairs-and-products}) as well as proofs of the algebraic
  properties of the latter (\cref{chapter:computation-laws-for-products}).
  Moreover, it provides common operations on maps (\cref{chapter:maps}), various
  properties of images and preimages (\cref{chapter:image-and-preimage}) and the
  notions of injectivity, surjectivity, bijectivity
  (\cref{chapter:injections-surjections-bijections}) and invertibility of maps
  (\cref{chapter:invertible-maps}).
  The library provides an axiom system characterizing sets (\cref{chapter:sets})
  and, furthermore, it covers the notions of binary relations
  (\cref{chapter:binary-relations}), fixed-points of subset preserving maps
  (\cref{chapter:fixed-points}), including and equinumerosity
  (\cref{chapter:equinumerosity}).

  As two famous results it includes the Knaster-Tarski fixed point theorem
  (\cref{FOUNDATIONS_12_8420450166112256}) and the Cantor-Schröder-Bernstein
  theorem (\cref{FOUNDATIONS_13_1913663275401216}).

  \paragraph*{Usage.}
  At the very beginning of each chapter you can find the name of its source
  file, e.g. \path{foundations/sections/01_classes.ftl.tex} for
  \cref{chapter:classes}. This filename can be used to import the chapter via
  \Naproche's \texttt{readtex} instruction to another ForTheL text, e.g.:
  \begin{center}
    \verb`[readtex \path{foundations/sections/01_classes.ftl.tex}]`
  \end{center}

  \paragraph*{Checking times.}
  The checking times for each of the chapters may vary from computer to
  computer, but on mid-range hardware they are likely to be similar to those
  given in table below:

  \begin{center}
    \begin{tabular}{c|c|c}

      & \multicolumn{2}{c}{\textbf{Checking time}}
      \\
      \textbf{Chapter}
      & \textbf{without dependencies}     & \textbf{with dependencies}
      \\ \hline
      \ref{chapter:classes}
      & 00:05 min                         & 00:05 min
      \\
      \ref{chapter:computation-laws-for-classes}
      & 00:10 min                         & 00:15 min
      \\
      \ref{chapter:symmetric-difference}
      & 00:30 min                         & 00:50 min
      \\
      \ref{chapter:pairs-and-products}
      & 00:10 min                         & 00:15 min
      \\
      \ref{chapter:computation-laws-for-products}
      & 01:35 min                         & 01:55 min
      \\
      \ref{chapter:maps}
      & 01:15 min                         & 01:25 min
      \\
      \ref{chapter:image-and-preimage}
      & 01:30 min                         & 02:55 min
      \\
      \ref{chapter:injections-surjections-bijections}
      & 00:40 min                         & 02:05 min
      \\
      \ref{chapter:invertible-maps}
      & 02:20 min                         & 04:25 min
      \\
      \ref{chapter:sets}
      & 02:15 min                         & 06:40 min
      \\
      \ref{chapter:binary-relations}
      & 00:15 min                         & 06:55 min
      \\
      \ref{chapter:fixed-points}
      & 00:35 min                         & 07:15 min
      \\
      \ref{chapter:equinumerosity}
      & 01:50 min                         & 09:00 min
    \end{tabular}
  \end{center}


  \subfile{sections/01_classes.ftl.tex}
  \subfile{sections/02_computation-laws-for-classes.ftl.tex}
  \subfile{sections/03_symmetric-difference.ftl.tex}
  \subfile{sections/04_pairs-and-products.ftl.tex}
  \subfile{sections/05_computation-laws-for-products.ftl.tex}
  \subfile{sections/06_maps.ftl.tex}
  \subfile{sections/07_image-and-preimage.ftl.tex}
  \subfile{sections/08_injections-surjections-bijections.ftl.tex}
  \subfile{sections/09_invertible-maps.ftl.tex}
  \subfile{sections/10_sets.ftl.tex}
  \subfile{sections/11_binary-relations.ftl.tex}
  \subfile{sections/12_fixed-points.ftl.tex}
  \subfile{sections/13_equinumerosity.ftl.tex}
\end{document}
