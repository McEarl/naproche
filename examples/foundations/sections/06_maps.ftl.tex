\documentclass[../../set-theory/set-theory.tex]{subfiles}

\begin{document}
  \chapter{Maps}\label{chapter:maps}

  \filename{foundations/sections/06_maps.ftl.tex}

  \begin{forthel}
    %[prove off][check off]

    [readtex \path{foundations/sections/04_pairs-and-products.ftl.tex}]

    %[prove on][check on]
  \end{forthel}


  \section{Ranges}

  \begin{forthel}
    \begin{definition}\printlabel{FOUNDATIONS_06_4284980337311744}
      Let $f$ be a map.
      A value of $f$ is an object $b$ such that $b = f(a)$ for some
      $a \in \dom(f)$.
    \end{definition}
  \end{forthel}

  \begin{forthel}
    \begin{definition}\printlabel{FOUNDATIONS_06_1938831225913344}
      Let $f$ be a map.
      The range of $f$ is
      \[ \{ f(a) \mid a \in \dom(f) \}. \]
    \end{definition}

    Let $\range{f}$ stand for the range of $f$.
  \end{forthel}

  \begin{forthel}
    \begin{proposition}\printlabel{FOUNDATIONS_06_6386349418479616}
      Let $f$ be a map and $b$ be an object.
      $b$ is a value of $f$ iff $b \in \range{f}$.
    \end{proposition}
    \begin{proof}
      Case $b$ is a value of $f$.
        Take $a \in \dom(f)$ such that $b = f(a)$.
        $b$ is an object.
        Hence $b \in \range{f}$.
      End.

      Case $b \in \range{f}$.
        Then $b$ is an object such that $b = f(a)$ for some $a \in \dom(f)$.
        Hence $b$ is a value of $f$.
      End.
    \end{proof}
  \end{forthel}


  \section{The identity map}

  \begin{forthel}
    \begin{definition}\printlabel{FOUNDATIONS_06_1920902360989696}
      Let $A$ be a class.
      % $\id{A}$ is the map defined on $A$ such that $\id{A}(a) = a$ for all
      % $a \in A$. %!
      $\id{A}$ is the map $h$ such that $h$ is defined on $A$ and $h(a) = a$
      for all $a \in A$.
    \end{definition}

    Let the identity map on $A$ stand for $\id{A}$.
  \end{forthel}


  \section{Composition}

  \begin{forthel}
    \begin{definition}\printlabel{FOUNDATIONS_06_7605717729017856}
      Let $f, g$ be maps.
      Assume $\range{f} \subseteq \dom(g)$.
      % $g \circ f$ is the map defined on $\dom(f)$ such that $(g \circ f)(a) =
      % g(f(a))$ for all $a \in \dom(f)$. %!
      $g \circ f$ is the map $h$ such that $h$ is defined on $\dom(f)$ and
      $h(a) = g(f(a))$ for all $a \in \dom(f)$.
    \end{definition}

    Let the composition of $g$ and $f$ stand for $g \circ f$.
  \end{forthel}


  \section{Restriction}

  \begin{forthel}
    \begin{definition}\printlabel{FOUNDATIONS_06_7095412741636096}
      Let $f$ be a map and $X \subseteq \dom(f)$.
      % $f \restriction X$ is the map defined on $X$ such that
      % $(f \restriction X)(a) = f(a)$ for all $a \in X$. %!
      $f \restriction X$ is the map $h$ such that $h$ is defined on $X$ and
      $h(a) = f(a)$ for all $a \in X$.
    \end{definition}

    Let the restriction of $f$ to $X$ stand for $f \restriction X$.
  \end{forthel}

  \begin{forthel}
    \begin{proposition}\printlabel{FOUNDATIONS_06_2170189258948608}
      Let $A$ be a class and $X \subseteq A$.
      Then $\id{A} \restriction X = \id{X}$.
    \end{proposition}
  \end{forthel}


  \section{Images and preimages}

  \begin{forthel}
    \begin{definition}\printlabel{FOUNDATIONS_06_3038237683613696}
      Let $f$ be a map and $A$ be a class.
      The image of $A$ under $f$ is
      \[ \{ f(a) \mid a \in \dom(f) \cap A \}. \]
    \end{definition}

    Let the direct image of $A$ under $f$ stand for the image of $A$ under $f$.
    Let $\image{f}(A)$ stand for the image of $A$ under $f$.
  \end{forthel}

  \begin{forthel}
    \begin{definition}\printlabel{FOUNDATIONS_06_4563167805964288}
      Let $f$ be a map and $B$ be a class.
      The preimage of $B$ under $f$ is
      \[ \{ a \in \dom(f) \mid f(a) \in B \}. \]
    \end{definition}

    Let the inverse image of $B$ under $f$ stand for the preimage of $B$ under
    $f$.
    Let $\preimage{f}(B)$ stand for the preimage of $B$ under $f$.
  \end{forthel}


  \section{Maps between classes}

  \begin{forthel}
    \begin{definition}\printlabel{FOUNDATIONS_06_6934038600220672}
      Let $A$ be a class.
      A map of $A$ is a map $f$ such that $\dom(f) = A$.
    \end{definition}
  \end{forthel}

  \begin{forthel}
    \begin{definition}\printlabel{FOUNDATIONS_06_7725375157174272}
      Let $B$ be a class.
      A map to $B$ is a map $f$ such that $f(a) \in B$ for each $a \in \dom(f)$.
    \end{definition}
  \end{forthel}

  \begin{forthel}
    \begin{definition}\printlabel{FOUNDATIONS_06_2823507398361088}
      Let $A, B$ be classes.
      A map from $A$ to $B$ is a map $f$ such that $\dom(f) = A$ and
      $f(a) \in B$ for each $a \in A$.
    \end{definition}

    Let $f: A \to B$ stand for $f$ is a map from $A$ to $B$.
  \end{forthel}

  \begin{forthel}
    \begin{definition}\printlabel{FOUNDATIONS_06_3390734908522496}
      Let $A$ be a class.
      A map on $A$ is a map from $A$ to $A$.
    \end{definition}
  \end{forthel}

  \begin{forthel}
    \begin{proposition}\printlabel{FOUNDATIONS_06_3312973569327104}
      Let $A, B$ be classes and $f, g : A \to B$.
      Assume that $f(a) = g(a)$ for all $a \in A$.
      Then $f = g$.
    \end{proposition}
  \end{forthel}

  \begin{forthel}
    \begin{proposition}
      Let $A, B$ be classes and $f$ be a map of $A$.
      Assume that $f(a) \in B$ for all $a \in A$.
      Then $f$ is a map from $A$ to $B$ iff $\range{f} \subseteq B$.
    \end{proposition}
  \end{forthel}

  \begin{forthel}
    \begin{proposition}\printlabel{FOUNDATIONS_06_5104361690628096}
      Let $A$ be a class.
      Then $\id{A}$ is a map on $A$.
    \end{proposition}
  \end{forthel}

  \begin{forthel}
    \begin{proposition}\printlabel{FOUNDATIONS_06_1706446651654144}
      Let $A, B, C$ be classes and $f : A \to B$ and $g : B \to C$.
      Then $g \circ f : A \to C$.
    \end{proposition}
  \end{forthel}

  \begin{forthel}
    \begin{proposition}\printlabel{FOUNDATIONS_06_4078561256275968}
      Let $A, B$ be classes and $f : A \to B$ and $X \subseteq A$.
      Then $f \restriction X : X \to B$.
    \end{proposition}
  \end{forthel}

  \begin{forthel}
    \begin{proposition}\printlabel{FOUNDATIONS_06_3964401904254976}
      Let $A, B$ be classes and $f : A \to B$.
      Then \[ f \circ \id{A} = f = \id{B} \circ f. \]
    \end{proposition}
    \begin{proof}
      $A$ is the domain of $f \circ \id{A}$ and the domain of $f$ and the
      domain of $\id{B} \circ f$.
      We have $(f \circ \id{A})(a)
        = f(\id{A}(a))
        = f(a)
        = \id{B}(f(a))
        = (\id{B} \circ f)(a)$
      for all $a \in A$.
      Hence $f \circ \id{A}
        = f
        = \id{B} \circ f$.
    \end{proof}
  \end{forthel}

  \begin{forthel}
    \begin{proposition}\printlabel{FOUNDATIONS_06_3118771061391360}
      Let $A$ be a class and $X \subseteq A$.
      Then \[ \id{A} \restriction X = \id{X}. \]
    \end{proposition}
    \begin{proof}
      We have $\dom(\id{A} \restriction X)
        = X
        = \dom(\id{X})$.
      $(\id{A} \restriction X)(a)
        = \id{A}(a)
        = a
        = \id{X}(a)$
      for all $a \in X$.
      Hence $\id{A} \restriction X = \id{X}$.
    \end{proof}
  \end{forthel}

  \begin{forthel}
    \begin{proposition}\printlabel{FOUNDATIONS_06_6866147389472768}
      Let $A, B, C, D$ be classes and $f : A \to B$ and $g : B \to C$ and
      $h : C \to D$.
      Then \[ h \circ (g \circ f) = (h \circ g) \circ f. \]
    \end{proposition}
    \begin{proof}
      $h \circ (g \circ f)$ and $(h \circ g) \circ f$ are maps from $A$ to $D$.

      Let us show that $(h \circ (g \circ f))(a) = ((h \circ g) \circ f)(a)$ for
      all $a \in A$.
        Let $a \in A$.
        Then $(h \circ (g \circ f))(a)
          = h((g \circ f)(a))
          = h(g(f(a)))
          = (h \circ g)(f(a))
          = ((h \circ g) \circ f)(a)$.
      End.

      Hence $h \circ (g \circ f) = (h \circ g) \circ f$.
    \end{proof}
  \end{forthel}


  \section{Maps and products}

  \begin{forthel}
    \begin{definition}\printlabel{FOUNDATIONS_06_5135634870042624}
      Let $f$ be a map such that $\dom(f) = A \times B$ for some nonempty
      classes $A, B$.
      Let $a$ be an object such that $(a,b) \in \dom(f)$ for some object $b$.
      $f(a,-)$ is the map such that $\dom(f(a,-)) = B$ and $f(a,-)(b) = f(a,b)$
      for all $b \in B$ where $B$ is the class such that $\dom(f) = A \times B$
      for some class $A$.
    \end{definition}
  \end{forthel}

  \begin{forthel}
    \begin{definition}\printlabel{FOUNDATIONS_06_3621991366000640}
      Let $f$ be a map such that $\dom(f) = A \times B$ for some nonempty
      classes $A, B$.
      Let $b$ be an object such that $(a,b) \in \dom(f)$ for some object $a$.
      $f(-,b)$ is the map such that $\dom(f(-,b)) = A$ and $f(-,b)(a) = f(a,b)$
      for all $a \in A$ where $A$ is the class such that $\dom(f) = A \times B$
      for some class $B$.
    \end{definition}
  \end{forthel}

  \begin{forthel}
    \begin{proposition}\printlabel{FOUNDATIONS_06_8946256734846976}
      Let $A, B, C$ be classes such that $A, B$ are nonempty and $a \in A$.
      Let $f$ be a map from $A \times B$ to $C$.
      Then $f(a,-)$ is a map from $B$ to $C$.
    \end{proposition}
  \end{forthel}

  \begin{forthel}
    \begin{proposition}\printlabel{FOUNDATIONS_06_8080207992848384}
      Let $A, B, C$ be classes such that $A, B$ are nonempty and $b \in B$.
      Let $f$ be a map from $A \times B$ to $C$.
      Then $f(-,b)$ is a map from $A$ to $C$.
    \end{proposition}
  \end{forthel}

  \begin{forthel}
    \begin{proposition}\printlabel{FOUNDATIONS_06_2754759509409792}
      Let $A, B, C$ be classes and $f$ be a map of $A \times B$.
      Assume that $f(a,b) \in C$ for all $a \in A$ and all $b \in B$.
      Then $f$ is a map from $A \times B$ to $C$.
    \end{proposition}
  \end{forthel}

  \begin{forthel}
    \begin{proposition}\printlabel{FOUNDATIONS_06_2304295212941312}
      Let $A, B, C$ be classes and $f$ be a map from $A \times B$ to $C$.
      Let $a \in A$ and $b \in B$.
      Then $f(a,b) \in C$.
    \end{proposition}
  \end{forthel}
\end{document}
