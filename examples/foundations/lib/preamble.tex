\usepackage{amssymb}
\usepackage{pgffor}
\usepackage{amstext}

\newcommand{\id}[1]{\mathrm{id}_{#1}}
\newcommand{\range}[1]{\mathrm{range}\left(#1\right)}
\newcommand{\symdiff}{\mathop{\triangle}}
\newcommand{\pow}[1]{\mathcal{P}\left(#1\right)}
\newcommand{\onto}{\twoheadrightarrow}
\newcommand{\into}{\hookrightarrow}
\newcommand{\image}[1]{\left(#1\right)_{*}}
\newcommand{\preimage}[1]{\left(#1\right)^{*}}
\newcommand{\inverse}[1]{\left(#1\right)^{-1}}
\newcommand{\funspace}[2]{[#1\to#2]}

% E.g. `\set{1, 2, 3, 4}` prints the numbers 1, 2, 3 and 4 as a comma
% separated list which is enclosed within curly braces.
\def\set#1{%
  \ensuremath{%
    \{%
    \foreach[count=\i] \x in {#1}{%
      \ifnum\i>1,\,\fi%
      \x%
    }%
    \}
  }%
}
