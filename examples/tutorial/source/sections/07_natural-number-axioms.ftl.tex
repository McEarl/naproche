\documentclass{stex}
\libinput{preamble}
\begin{document}

\section{Postulating Natural Number Axioms}

We need to introduce axioms for our abstract first-order structure.
Axioms are ForTheL statements written in axiom environments.
For arithmetic we use self-explanatory symbolic formulas.
There are many ways of axiomatizing the natural numbers in order
to be able to prove our final goal: the infinitude of
primes. Here we axiomatize the natural numbers as
a sort of commutative ``half-ring'' with $1$.
We provide ourselves with another variable $l$,
pretyped as a natural number.

\begin{forthel}
\begin{axiom} $m + n = n + m$.
\end{axiom}

Let $l$ stand for a natural number.

\begin{axiom} $(m + n) + l = m + (n + l)$.
\end{axiom}

\begin{axiom}  $m + 0 = m = 0 + m$.
\end{axiom}

\begin{axiom} $m * n = n * m$.
\end{axiom}

\begin{axiom} $(m * n) * l = m * (n * l)$.
\end{axiom}

\begin{axiom} $m * 1 = m = 1 * m$.
\end{axiom}

\begin{axiom} $m * 0 = 0 = 0 * m$.
\end{axiom}

\begin{axiom} $m * (n + l) = (m * n) + (m * l)$ and
                $(n + l) * m = (n * m) + (l * m)$.
\end{axiom}

\begin{axiom} If $l + m = l + n$ or $m + l = n + l$
then $m = n$.
\end{axiom}

\begin{axiom} Assume that $l$ is nonzero.
If $l * m = l * n$ or $m * l = n * l$ then $m = n$.
\end{axiom}

\begin{axiom} If $m + n = 0$ then $m = 0$ and $n = 0$.
\end{axiom}

\end{forthel}

Axioms - like Signatures - are toplevel sections which consist of
$n + 1$ statements. The first $n$ are assumption statements
(``Assume ...'', ``Let ...'')
under which the final statement is postulated. Note that
pretypings of variables also act like assumptions.

\end{document}
