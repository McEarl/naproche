\documentclass{stex}
\libinput{preamble}
\begin{document}

\section{An Interactive Proof}

The final contradiction in Euclid's proof will need:

\begin{lemma} Let $l | m$ and $l | m + n$. Then $l | n$.
\end{lemma}

\Naproche{} does not find
a proof on its own: depending on some default timeouts the
proof search is abandoned, and the goal $l | n$ fails.
In Isabelle-Naproche this is signaled in the output
window, and the failed goal is underlined in red.

So the user has to ``interactively'' supply a proof, which in a first
approximation is a list
of statements leading up to the claim, and which
\Naproche{}'s ATP is able to prove successively.
Proof statements can also introduce assumptions
and new variables to the argument, and they can
structure the proof.

\begin{forthel}
\begin{lemma} Let $l | m$ and $l | m + n$. Then $l | n$.
\end{lemma}

\begin{proof}
Assume that $l$ is nonzero.
Take a natural number $p$ such that $m = l * p$.
Take a natural number $q$ such that $m + n = l * q$.

Let us show that
$p \leq q$.
Proof by contradiction.
Assume the contrary. Then $q < p$.
$m+n = l * q < l * p = m$.
Contradiction. qed.

Take $r = q - p$.
We have $(l * p) + (l * r) = l * q = m + n = (l * p) + n$.
Hence $n = l * r$.
\end{proof}

\end{forthel}

When \Naproche{} encounters a statement immediately followed by an
explicit proof
then \Naproche{} defers proving the statement and first goes through
the proof. Since proofs may contain subproofs, this process may
take place recursively.

Proofs of a ``toplevel'' Lemma or Theorem use the

\verb+\begin{proof}...\end{proof}+

\noindent environment well-known from \LaTeX.
In our proof there is also a ``lowlevel'' proof of $p \leq q$
indicated by ``Let us show that''. We discuss some aspects of the
proof:
\begin{itemize}
\item Most sentences in a proof are statements, or statements
extended by constructs that organize the flow of the argument.
\item ``Assume that $l$ is nonzero.'' is an assumption that introduces
the premise ``$l$ is nonzero'' to the argument. Instead of ``Assume that''
one could also use variants like
[let us $\mid$ we can] (assume $\mid$ presume $\mid$ suppose) [that].
\item ``Take $p$ such that $m = l * p$.'' introduces a new variable
$p$ with a specific property to the argument. To verify this construct
the prover has to show the existence of some object satisfying the
property. Again there are variants:
[let us $\mid$ we can] (choose $\mid$ take $\mid$ consider).
\item ``Let us show that $p \leq q$.'' claims that the statement
$p \leq q$ holds and announces a subsequent proof.
Alternatives: [let us $\mid$ we can]
(prove $\mid$ show $\mid$ demonstrate) (that).
\item ``Proof by contradiction'' denotes the start of an indirect
proof. It is recommended to explicitly mark indirect proofs.
Note that in the example there is a ``lowlevel'' proof that
uses a simple

\verb+Proof [by ...](.) ... (qed. | end.)+

\noindent environment instead of the \LaTeX proof environment.
\item Other proof methods are ``by cases'' and ``by induction''.
\item ``Assume the contrary.'': The contrary is the negation
of the current thesis which in this case is the statement claimed
just before. ``thesis'' denotes the current thesis, ``contradiction''
stands for ``false''.
\item ``Then $q < p$.'': Words like ``then'', ``hence'', ``thus'',
``therefore'', ``consequently'' are filler words which are redundant
for \Naproche{} but may help human readers to understand the text.
\item ``$m+n = l * q < l * p = m$'': binary relations like ``$=$''
or ``$<$'' can be chained. The statement means the conjunction
of the single relations. These will be checked from
left to right.
\item ``Contradiction. qed.'': The indirect proof has reached the
desired contradiction, and the (local) proof environment is closed by
``qed.''.
\end{itemize}

\Naproche{} is able to prove the next lemma without an
explicit proof in the text.

\begin{forthel}

\begin{lemma} Let $m | n \neq 0$. Then $m \leq n$.
\end{lemma}
\end{forthel}

\end{document}
