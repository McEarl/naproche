\documentclass{stex}
\libinput{preamble}
\begin{document}

\section{Primes}

Prime numbers are defined as usual. Indeed we define
the adjective ``prime'' which will enable us
to write ``prime natural number'' or ``prime divisor''.

\begin{forthel}

Let $x$ is nontrivial stand for $x \neq 0$ and $x \neq 1$.

\begin{definition}
$n$ is prime iff $n$ is nontrivial and
    for every divisor $m$ of $n$ $m = 1$ or $m = n$.
\end{definition}

\end{forthel}

The following lemma obviously holds by induction: either
$k$ is prime itself, or $k$ has a divisor strictly
between $1$ and $k$; by induction that divisor has a prime
divisor which is also a prime divisor of $k$.

\begin{forthel}
%[prove off]
\begin{lemma} Every nontrivial $m$ has a prime divisor.
\end{lemma}
\begin{proof}[by induction on $m$] \end{proof}
\end{forthel}


``Proof by induction'' transforms the thesis into the inductivity
theses:
\begin{footnotesize}
\begin{verbatim}
thesis: forall v0 ((aNaturalNumber(v0) and (not v0 = 0 and not v0 = 1))
implies ((InductionHypothesis :: forall v1 ((aNaturalNumber(v1) and
(not v1 = 0 and not v1 = 1)) implies (iLess(v1,v0)
implies exists v2 ((aNaturalNumber(v2) and doDivides(v2,v1))
and isPrime(v2))))) implies exists v1
((aNaturalNumber(v1) and doDivides(v1,v0)) and isPrime(v1))))
\end{verbatim}
\end{footnotesize}
which can be discharged automatically by eprover.

\end{document}
