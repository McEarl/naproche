\documentclass{stex}
\libinput{preamble}
\begin{document}

\section{Importing General Preliminaries}

The \Naproche{} system comes along with a small inbuilt language which
includes the notions of
(mathematical) object, classes, sets, maps and functions:
\begin{quotation}
Elements of classes or sets are objects, maps and functions map objects
to objects. Sets are classes which are objects.
Similarly functions are maps which are objects.
\end{quotation}
Basic properties can be checked in the system. E.g.,
\begin{greybox}
\begin{theorem} Every element of every class is an object. \end{theorem}
\begin{proof} Let $C$ be a class. Let $x$ be an element of $C$.
Then $x$ is an object.
\end{proof}
\end{greybox}

Note that the inbuilt premisses are not available to the external prover
but only to the inbuilt \textit{reasoner} of \Naproche. Since the reasoner is
rather weak, we have to give an explicit proof: instantiate the universal
quantifiers with ``arbitrary but fixed'' instances $C$ and $x$ and prove
the claim in this context. Note that we use the \LaTeX{} proof environment
also as ForTheL proof environment.

The file \path{preliminaries.ftl.tex} proves important properties of the
inbuilt notions, similar to the above example, and it postulates further
axioms about them. More details can be found in the commentary parts of the file.
The file also includes a list \path{vocabulary.ftl} of singular/plural pairs
like the above
\verb+[synonym number/numbers]+, to be used for grammatical correctness.
Moreover, some alternative phrases for certain mathematical phrases are
imported with \path{macros.ftl}.

These preliminaries are imported by:
\begin{forthel}
[readtex preliminaries/source/preliminaries.ftl.tex]
\end{forthel}

This file also provides the $\subseteq$-relation between classes, so that
we can prove:
\begin{forthel}
\begin{theorem}
Let $C$ be a class. Then $C \subseteq C$.
\end{theorem}
\end{forthel}

\begin{exercise}
Prove that $\subseteq$ is a transitive relation, and that
the empty set is a subclass of every other class.
One can also denote the empty set symbolically by
using the linguistic command:\\
Let $\emptyset$ denote the empty set.
\end{exercise}

\begin{exercise}
The preliminaries file also provides the notion $F(x)$ for elements $x \in \dom(F)$
where $F$ is a function or map.
Formalize the property that a function is injective and that
the composition of two injective functions is injective.
\end{exercise}

\end{document}
