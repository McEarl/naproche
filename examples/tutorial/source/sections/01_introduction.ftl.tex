\documentclass{stex}
\libinput{preamble}
\begin{document}

\section{Introduction}

\Naproche{} (Natural Proof Checking) is a mathematical proof assistant
for texts in the input language ForTheL (Formula Theory Language).
ForThel aims to approximate the
language, structure and appearance of common mathematical texts;
FortheL texts are checked by \Naproche{} for syntactic, ontological
and logical correctness. \Naproche{} is a component of
Isabelle PIDE (Proof Interactive Development Environment) which provides
comfortable text editing and interactive checking through a continuously
running \Naproche{} server.

Ideally, a user edits a mathematical text in \LaTeX{}, and
\Naproche{} automatically checks the correctness of those parts which are
in \verb+\begin+\verb+{forthel}+ ... \verb+\end{forthel}+
environments and gives
feedback to the user, similar to a continuous spellchecker.
Although that goal is still far away, the current \Naproche{} offers
the possibility to write university-level mathematical texts
in a natural language and style whose ForTheL segments are logically
verified by \Naproche{}. In this tutorial and in example texts
which are distributed together with \Naproche{} these segments
are typeset on a grey background.

\Naproche{} is a derivative of the ground-breaking System
for Automated Deduction (SAD). \Naproche{} still supports the
ASCII format \path{.ftl} of SAD which allows rapid
experiments without worrying about LaTeX particulars, and we
use that format for a Quick Start and some exercises.

This tutorial is an introduction to the principles and
use of the \Naproche{} prover. After the Quick Start and
some general information we explain
ForTheL commands and statements and the structuring of ForTheL texts
along a proof of the infinitude of primes. Actually the tutorial is
a version of Euclid's proof, preceded by
necessary number-theoretic and set-theoretic preliminaries.
After studying this material, a reader should be able to understand the
other example texts and start writing similar formalizations.
Let us, however, warn the reader that writing formally complete
and unambiguous mathematical texts in a readable natural language
is subtle and requires ample exercise, as one may already guess by
looking at legal texts.

We suggest that the tutorial is read as pdf and simultaneously followed
interactively in the Isabelle
PIDE (Proof Interactive Development Environment), with the source
\path{TUTORIAL.ftl.tex} opened in one buffer, using other buffers
for experiments and exercises. Moreover one may use a \LaTeX{} compiler
for displaying \path{.ftl.tex} formalizations.

\end{document}
