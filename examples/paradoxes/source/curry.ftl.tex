\documentclass{stex}
\libinput{preamble}
\newcommand{\var}[1]{\mathrm{v}_{#1}}
\newcommand{\abs}[2]{\lambda\var{#1}.\ #2}
\newcommand{\app}[2]{(#1)(#2)}
\newcommand{\fix}{\mathtt{fix}}
\begin{document}
\title{Curry's Paradox in \Naproche}
\author{Marcel Schütz}
\date{2023}
\maketitle
\begin{smodule}{curry}
\usemodule{mod?haskell-curry}
\usemodule{mod?curry-paradox}
\begin{sparagraph}
  \noindent \emph{\sn{Curry's Paradox}} is a paradox described by \sr{Curry}{Haskell Curry} in 1942 \cite{Curry1942}.
  It allows the derivation of an arbitrary statement from a self-referential expression that presupposes its own validity.
\end{sparagraph}

\section*{Untyped $\lambda$-calculus}

\begin{forthel}
  %[prove off][check off]

  [readtex \path{arithmetic/sections/01_natural-numbers.ftl.tex}]

  %[prove on][check on]
\end{forthel}

\begin{sparagraph}
  \noindent Our formalization of \sn{Curry's Paradox} in \Naproche is based on an untyped $\lambda$-calculus:
\end{sparagraph}

\begin{forthel}
  \begin{signature*}
    A expression is a notion.
  \end{signature*}

  Let $A, B$ denote expressions.

  \begin{signature*}
    A variable is an expression.
  \end{signature*}

  Let $n$ denote a natural number.

  \begin{signature*}
    $\var{n}$ is a variable.
  \end{signature*}

  \begin{signature*}[Abstraction]\label{abstraction}
    $\abs{n}{A}$ is an expression.
  \end{signature*}

  \begin{signature*}[Application]\label{application}
    $\app{A}{B}$ is an expression.
  \end{signature*}

  \begin{signature*}[Fixed-point combinator]\label{fixed_point_combinator}
    $\fix$ is an expression such that
    \[\app{\fix}{A} = \app{A}{\app{\fix}{A}}\]
    for every expression $A$.
  \end{signature*}
\end{forthel}


\section*{Propositional logic}

\begin{sparagraph}
  \noindent We extend this $\lambda$-calculus by a fragment of propositional logic:
\end{sparagraph}

\begin{forthel}
  \begin{signature*}[Implication]\label{implication}
    $A \rightarrow B$ is an expression.
  \end{signature*}

  \begin{signature*}\label{truth}
    $A$ is true is a relation.
  \end{signature*}

  \begin{axiom}\label{beta_reduction}
    $\app{\abs{n}{\var{n} \rightarrow B}}{A} = A \rightarrow B$.
  \end{axiom}

  \begin{axiom}\label{reflexivity}
    $A \rightarrow A$ is true.
  \end{axiom}

  \begin{axiom}\label{modus_ponens}
    If $A$ is true and $A \rightarrow B$ is true then $B$ is true.
  \end{axiom}

  \begin{axiom}\label{strengthening}
    If $A \rightarrow (A \rightarrow B)$ is true then $A \rightarrow B$ is true.
  \end{axiom}
\end{forthel}


\section*{Curry's paradox}

\begin{sparagraph}
  \noindent Using the fixpoint combinator from above we can formulate a self-referential expression $X$ of the form “If $X$ is true then $A$ is true” for any arbitrary expression $A$ by defining $X = \app{\fix}{\abs{0}{\var{0} \rightarrow A}}$.
  From the existence of such an expression $X$ together with the above axioms we can then derive that any expression $A$ is true.
\end{sparagraph}

\begin{forthel}
  \begin{theorem*}[Curry's paradox]\label{curry_paradox}
    Every expression is true.
  \end{theorem*}
  \begin{proof}
    Let $A$ be an expression.
    Take $N = \abs{0}{\var{0} \rightarrow A}$ and $X = \app{\fix}{N}.$

    (1) Then $X = \app{N}{X} = X \rightarrow A$ (by \cref{beta_reduction}).

    Hence $X \rightarrow (X \rightarrow A)$ is true (by 1, \cref{reflexivity}).

    (2) Thus $X \rightarrow A$ is true (by \cref{strengthening}).

    (3) Therefore $X$ is true (by 1, 2).

    Consequently $A$ is true (by \cref{modus_ponens}, 2, 3).
  \end{proof}
\end{forthel}
\end{smodule}
\printbibliography
\end{document}
