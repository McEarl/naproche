\documentclass{stex}
\libinput{preamble}
\libinput[naproche/examples/foundations]{preamble}

\title{Hilbert's Paradox in \Naproche}
\author{Marcel Schütz}
\date{2023}

\begin{document}
\maketitle

\noindent \emph{Hilbert's Paradox}, discovered around 1900 by David Hilbert, demonstates that there cannot exists a set that is closed under powerset and under union of arbitrary subset \cite{PeckhausKahl2002}.

\inputref[naproche/examples/set-theory]{thm/hilbert-paradox.ftl.tex}

\noindent Using Hilbert's Paradox it can further be shown that there exists no universal set.

\begin{fcorollary*}
  There exists no set that contains all sets.
\end{fcorollary*}
\begin{fproof}[method=contradiction]
  Assume the contrary.
  Consider a set $X$ that contains all sets.
  Define $V = \fclass{x \in X}{\text{$x$ is a set}}$.
  Then $V$ is closed under powersets and closed under unions.
  Hence $V$ is not a set (by \printref{hilbert_paradox}).
\end{fproof}

\printbibliography
\end{document}
