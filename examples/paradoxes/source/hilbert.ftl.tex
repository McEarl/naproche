\documentclass{stex}
\libinput{preamble}
\libinput[naproche/examples/foundations]{preamble}

\title{Hilbert's Paradox in \Naproche}
\author{Marcel Schütz}
\date{2023}

\begin{document}
\maketitle

\noindent \emph{Hilbert's Paradox}, discovered around 1900 by David Hilbert, demonstates that there cannot exists a set that is closed under powerset and under union of arbitrary subset \cite{PeckhausKahl2002}.

\inputref[naproche/examples/paradoxes]{mod/hilbert-paradox.ftl.tex}

\noindent Using Hilbert's Paradox it can further be shown that there exists no universal set.

\inputref[naproche/examples/paradoxes]{mod/hilbert-no-universal-set.ftl.tex}

\printbibliography
\end{document}
