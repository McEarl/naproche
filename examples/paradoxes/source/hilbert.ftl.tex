\documentclass{stex}
\libinput{preamble}
\libinput[naproche/examples/foundations]{preamble}
\begin{document}
\title{Hilbert's Paradox in \Naproche}
\author{Marcel Schütz}
\date{2023}
\maketitle
\begin{smodule}{hilbert}
\usemodule{mod?david-hilbert}
\usemodule{mod?hilbert-paradox}
\begin{sparagraph}
  \noindent \emph{\sn{Hilbert's Paradox}}, discovered around 1900 by \sr{Hilbert}{David Hilbert}, demonstates that there cannot exists a set that is closed under powerset and under union of arbitrary subset \cite{PeckhausKahl2002}.
\end{sparagraph}

\begin{forthel}
  %[prove off][check off]
  [readtex \path{foundations/sections/10_sets.ftl.tex}]
  %[prove on][check on]

  \begin{definition*}
    Let $S$ be a system of sets.
    $S$ is closed under powersets iff $\pow{x} \in S$ for all $x \in S$.
  \end{definition*}

  \begin{definition*}
    Let $S$ be a system of sets.
    $S$ is closed under unions iff $\bigcup X \in S$ for all $X \subseteq S$.
  \end{definition*}

  \begin{theorem*}[Hilbert's Paradox]\label{hilbert_paradox}
    There exists no system of sets that is closed under powersets and closed under unions.
  \end{theorem*}
  \begin{proof}[ by contradiction]
    Assume the contrary.
    Consider a system of sets $S$ that is closed under powersets and closed under unions.
    We have $S \subseteq S$.
    Hence $\bigcup S \in S$.
    Thus $\pow{\bigcup S} \in S$.
    Contradiction.
  \end{proof}
\end{forthel}

\begin{sparagraph}
  \noindent Using \sn{Hilbert's Paradox} it can further be shown that there exists no universal set.
\end{sparagraph}

\begin{forthel}
  \begin{corollary*}
    There exists no set that contains all sets.
  \end{corollary*}
  \begin{proof}[ by contradiction]
    Assume the contrary.
    Consider a set $X$ that contains all sets.
    Define $V = \{ x \in X \mid x$ is a set$\}$.
    Then $V$ is closed under powersets and closed under unions.
    Hence $V$ is not a set (by \nameref{hilbert_paradox}).
  \end{proof}
\end{forthel}
\end{smodule}
\printbibliography
\end{document}
