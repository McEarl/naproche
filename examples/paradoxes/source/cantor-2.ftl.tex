\documentclass{stex}
\libinput{preamble}
\libinput[naproche/examples/foundations]{preamble}
\libinput[naproche/examples/set-theory]{preamble}
\usepackage{xr}
\externaldocument{../../set-theory/set-theory}
\begin{document}
\title{Cantor's Second Paradox in \Naproche}
\author{Marcel Schütz}
\date{2023}
\maketitle
\begin{smodule}{cantor-2.ftl}
  \noindent \emph{Cantor's Second Paradox} denotes the observation that the collection of all set cannot be a set itself.
  It was shown by Georg Cantor in 1899 via his famous theorem stating that the cardinality of any set is strictly smaller than the cardinality of its powerset \cite[chapter 163]{Cantor1991}.

  \begin{forthel}
    %[prove off][check off]
    [readtex \path{set-theory/sections/06_cardinals.ftl.tex}]
    %[prove on][check on]
  \end{forthel}

  \begin{forthel}
    \begin{theorem*}[Cantor's Second Paradox]\label{cantor_paradox_2}
      There exists no set that contains all sets.
    \end{theorem*}
    \begin{proof}
      Assume the contrary.
      Consider a set $V$ that contains all sets.
      Then $\pow{V}$ is a set.
      Hence $\pow{V}$ is a subset of $V$.
      Thus $\card{\pow{V}} \leq \card{V}$.
      Contradiction.
      Indeed $\card{x} \less \card{\pow{x}}$ for any set $x$ (by \nameref{SET_THEORY_06_914271456198656}).
    \end{proof}
  \end{forthel}
\end{smodule}
\printbibliography
\end{document}
