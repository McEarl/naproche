\documentclass{stex}
\libinput{preamble}
\begin{document}
\title{Burali-Forti's Paradox in \Naproche}
\author{Marcel Schütz}
\date{2023}
\maketitle
\begin{smodule}{burali-forti.ftl}
  \noindent \emph{Burali-Forti's Paradox}, named after Cesare Burali-Forti, demonstates that there cannot exists a set that contains all ordinal number \cite{BuraliForti1897}.

  \begin{forthel}
    %[prove off][check off]
    [readtex \path{set-theory/sections/02_ordinals.ftl.tex}]
    %[prove on][check on]
  \end{forthel}

  \begin{forthel}
    \begin{theorem*}[Burali-Forti's Paradox]\label{burali_forti_paradox}
      There exists no set that contains all ordinals.
    \end{theorem*}
    \begin{proof}[ by contradiction]
      Assume the contrary.
      Consider a set $O$ that contains all ordinals.
      Then $O$ is transitive and every element of $O$ is transitive.
      Hence $O$ is an ordinal.
      Thus $O \in O$.
      Contradiction.
    \end{proof}
  \end{forthel}
\end{smodule}
\printbibliography
\end{document}
