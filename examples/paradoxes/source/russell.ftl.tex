\documentclass{stex}
\libinput{preamble}
\begin{document}
\title{Russell's Paradox in \Naproche}
\author{Marcel Schütz}
\date{2023}
\maketitle
\begin{smodule}{russell.ftl}
  \importmodule[naproche/examples/preliminaries]{everyday-ontology.ftl}

  \vardef{phivar}[args=1]{\maincomp{\varphi}\dobrackets{#1}}
  \vardef{xvar}{x}

  \noindent \emph{Russell's Paradox} is a set-theoretical paradox discovered by Bertrand Russell around 1902 \cite[chapter XV]{Frege1980} which shows that there exist statements $\phivar!$ whose extension $\class{\xvar}{\phivar{\xvar}}$ cannot constitute a set – or in other words: Not every class is a set.

  \begin{forthel}
    \begin{theorem*}[Russell's Paradox]\label{russell_paradox}
      It is wrong that every class is a set.
    \end{theorem*}
    \begin{proof}
      Assume the contrary.
      Define \[R = \class{x}{\text{$x$ is a set such that $x \notin x$}}.\]
      Then $R$ is a set.
      Hence $R \in R$ iff $R \notin R$.
      Contradiction.
    \end{proof}
  \end{forthel}
\end{smodule}
\printbibliography
\end{document}
