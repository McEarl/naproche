\documentclass{stex}
\libinput{preamble}

\title{Russell's Paradox in \Naproche}
\author{Marcel Schütz}
\date{2023}

\begin{document}
\maketitle

\vardef{phivar}[args=1]{\maincomp{\varphi}\dobrackets{#1}}
\vardef{xvar}{x}

\noindent \emph{Russell's Paradox} is a set-theoretical paradox discovered by Bertrand Russell around 1902 \cite[chapter XV]{Frege1980} which shows that there exist statements $\phivar!$ whose extension $\fclass{\xvar}{\phivar{\xvar}}$ cannot constitute a set – or in other words: Not every class is a set.

\inputref[naproche/examples/paradoxes]{mod/russell-paradox.ftl.tex}

\printbibliography
\end{document}
