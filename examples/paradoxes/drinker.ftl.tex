\documentclass[english]{article}
\usepackage[type={CC},modifier={zero},version={1.0},imagemodifier=-80x15]{doclicense}
\usepackage{biblatex}
\usepackage{../../lib/tex/naproche}
\addbibresource{REFERENCES.bib}
\begin{document}
\title{The Drinker Paradox in \Naproche}
\author{Marcel Schütz}
\date{2023}
\pagenumbering{gobble}
\maketitle

\noindent The \href{https://en.wikipedia.org/wiki/Drinker_paradox}{\emph{Drinker Paradox}} is a principle of \href{https://en.wikipedia.org/wiki/Classical_logic}{classical} \href{https://en.wikipedia.org/wiki/First-order_logic}{predicate logic} popularised by the logician \href{https://en.wikipedia.org/wiki/Raymond_Smullyan}{Raymond Smullyan} in his 1978 book \textit{What Is the Name of this Book?} \cite{Smullyan1978} which can be stated as:

\begin{quotation}
  \noindent There is someone in the pub such that, if he is drinking then everyone in the pub is drinking.
\end{quotation}

\begin{forthel}
  [readtex \path{paradoxes/ONTOLOGY.ftl.tex}]

  \begin{signature*}
    The pub is an object.
  \end{signature*}

  \begin{theorem*}[Drinker Paradox]\label{drinker_paradox}
    Assume that there is a person inside the pub.
    Then there is a person $P$ inside the pub such that if $P$ is drinking then every person inside the pub is drinking.
  \end{theorem*}
  \begin{proof}[by case analysis]
    Case every person inside the pub is drinking.
      Choose a person $P$ inside the pub.
      Then $P$ is drinking and every person inside the pub is drinking.
      Hence if $P$ is drinking then every person inside the pub is drinking.
    End.

    Case there is a person inside the pub that is not drinking.
      Consider a person $P$ inside the pub that is not drinking.
      Then if $P$ is drinking then every person inside the pub is drinking.
    End.
  \end{proof}
\end{forthel}

\printbibliography
\vfill
\doclicenseThis
\end{document}
