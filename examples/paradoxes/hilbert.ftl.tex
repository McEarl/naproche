\documentclass[english]{article}
\usepackage[type={CC},modifier={zero},version={1.0},imagemodifier=-80x15]{doclicense}
\usepackage{biblatex}
\usepackage[foundations]{../../lib/tex/naproche}
\addbibresource{REFERENCES.bib}
\begin{document}
\title{Hilbert's Paradox in \Naproche}
\author{Marcel Schütz}
\date{2023}
\pagenumbering{gobble}
\maketitle

\noindent \href{https://en.wikipedia.org/wiki/Von_Neumann_universe#Hilbert's_paradox}{\emph{Hilbert's paradox}}, discovered around 1900 by \href{https://en.wikipedia.org/wiki/David_Hilbert}{David Hilbert}, demonstates that there cannot exists a \href{https://en.wikipedia.org/wiki/Set_(mathematics)}{set} that is closed under \href{https://en.wikipedia.org/wiki/Power_set}{powersets} and under \href{https://en.wikipedia.org/wiki/Union_(set_theory)}{unions} of arbitrary \href{https://en.wikipedia.org/wiki/Subset}{subsets} \cite{PeckhausKahl2002}.

\begin{forthel}
  %[prove off][check off]
  [readtex \path{foundations/sections/10_sets.ftl.tex}]
  %[prove on][check on]

  \begin{definition*}
    Let $S$ be a system of sets.
    $S$ is closed under powersets iff $\pow(x) \in S$ for all $x \in S$.
  \end{definition*}

  \begin{definition*}
    Let $S$ be a system of sets.
    $S$ is closed under unions iff $\bigcup X \in S$ for all $X \subseteq S$.
  \end{definition*}

  \begin{theorem*}[Hilbert's Paradox]\label{hilbert_paradox}
    There exists no system of sets that is closed under powersets and closed under unions.
  \end{theorem*}
  \begin{proof}[by contradiction]
    Assume the contrary.
    Consider a system of sets $S$ that is closed under powersets and closed under unions.
    We have $S \subseteq S$.
    Hence $\bigcup S \in S$.
    Thus $\pow(\bigcup S) \in S$.
    Contradiction.
  \end{proof}
\end{forthel}

\noindent Using \href{https://en.wikipedia.org/wiki/Von_Neumann_universe#Hilbert's_paradox}{Hilbert's paradox} it can further be shown that there exists no \href{https://en.wikipedia.org/wiki/Universal_set}{universal set}.

\begin{forthel}
  \begin{corollary*}
    There exists no set that contains all sets.
  \end{corollary*}
  \begin{proof}[by contradiction]
    Assume the contrary.
    Consider a set $X$ that contains all sets.
    Define $V = \{ x \in X \mid x$ is a set$\}$.
    Then $V$ is closed under powersets and closed under unions.
    Hence $V$ is not a set (by \nameref{hilbert_paradox}).
  \end{proof}
\end{forthel}

\printbibliography
\vfill
\doclicenseThis
\end{document}
