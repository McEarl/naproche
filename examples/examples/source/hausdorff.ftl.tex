\documentclass{stex}
\libinput{preamble}
\libinput[naproche/examples/foundations]{preamble}
\libinput[naproche/examples/set-theory]{preamble}

\newcommand{\surjects}{\twoheadrightarrow}
\newcommand{\constzero}[1]{0^{#1}}
\newcommand{\cardsucc}[1]{\left(#1\right)^{+}}

\title{Regularity of successor cardinals}
\author{\Naproche formalization: \vspace{0.5em} \\
Steffen Frerix (2018), \\
Adrian De Lon (2021)}
\date{}

\begin{document}
  \maketitle

  \noindent This is a formalization of a theorem of Felix Hausdorff stating that
  successor cardinals are always regular.
  On mid-range hardware \Naproche needs approximately 1:30 Minutes to verify it,
  plus approximately 25 minutes to verify the library files it depends on.


  \section{Preliminaries}

  \begin{forthel}
    %[prove off][check off]

    [readtex \path{set-theory/sections/06_cardinals.ftl.tex}]

    %[prove on][check on]
  \end{forthel}

  \begin{forthel}
    Let $X$ denote a set.
    Let $\kappa$ denote a cardinal.

    \begin{axiom*}
      $\card{X \times X} = \card{X}$.
    \end{axiom*}

    \begin{signature*}
      $\cardsucc{\kappa}$ is a cardinal such that $\kappa \less \cardsucc{\kappa}$ and there is no
      cardinal $\nu$ such that $\kappa \less \nu \less \cardsucc{\kappa}$.
    \end{signature*}

    \begin{axiom*}
      $\card{\alpha} \leq \kappa$ for every element $\alpha$ of $\cardsucc{\kappa}$.
    \end{axiom*}

    \begin{definition*}
      The constant zero on $X$ is the function $f$ such that $\dom(f) = X$ and
      $f(x) = 0$ for every $x \in X$.
    \end{definition*}

    Let $\constzero{X}$ stand for the constant zero on $X$.
  \end{forthel}


  \section{Cofinality and regular cardinals}

  \begin{forthel}
    \begin{definition*}[Cofinality]
      Let $Y$ be a subset of $\kappa$.
      $Y$ is cofinal in $\kappa$ iff for every element $x$ of $\kappa$ there
      exists an element $y$ of $Y$ such that $x \less y$.
    \end{definition*}

    Let a cofinal subset of $\kappa$ stand for a subset of $\kappa$ that is
    cofinal in $\kappa$.

    \begin{definition*}
      $\kappa$ is regular iff $\card{x} = \kappa$ for every cofinal subset $x$ of
      $\kappa$.
    \end{definition*}
  \end{forthel}


  \section{Hausdorff's theorem}

  The following result appears in \cite[p.~443]{Hausdorff1908},
  where Hausdorff mentions that the proof is
  \textit{``ganz einfach''} (\textit{``very simple''}) and can be skipped.

  \begin{forthel}
    \begin{theorem*}[Hausdorff]
      $\cardsucc{\kappa}$ is regular.
    \end{theorem*}
    \begin{proof}[by contradiction]
      Assume the contrary.
      Take a cofinal subset $x$ of $\cardsucc{\kappa}$ such that $\card{x} \neq \cardsucc{\kappa}$.
      Then $\card{x} \leq \kappa$.
      Take a surjective map $f$ from $\kappa$ onto $x$
      (by \printref{SET_THEORY_06_192336220913664}).
      Indeed $x$ and $\kappa$ are nonempty and $\card{\kappa} = \kappa$.
      Then $f(\xi) \in \cardsucc{\kappa}$ for all $\xi \in \kappa$.
      For all $z \in \cardsucc{\kappa}$ if $z$ has an element then there exists a
      surjective map $h$ from $\kappa$ onto $z$.
      Indeed $\kappa$ is nonempty.

      Define \[ g(z) =
        \begin{cases}
          \text{choose $h : \kappa \onto z$ in $h$}
          & : \text{$z$ has an element}
          \\
          \text{$\constzero{\kappa}$}
          & : \text{$z$ has no element}
        \end{cases}
      \] for $z$ in $\cardsucc{\kappa}$.

      Let us show that for all $\xi, \zeta \in \kappa$ $g(f(\xi))$ is a map such
      that $\zeta \in \dom(g(f(\xi)))$.
        Let $\xi, \zeta \in \kappa$.
        If $f(\xi)$ has an element then $g(f(\xi))$ is a surjective map from
        $\kappa$ onto $f(\xi)$.
        If $f(\xi)$ has no element then $g(f(\xi)) = \constzero{\kappa}$.
        Hence $\dom(g(f(\xi))) = \kappa$.
        Therefore $\zeta \in \dom(g(f(\xi)))$.
      End.

      For all objects $\xi, \zeta$ we have $\xi, \zeta \in \kappa$ iff
      $(\xi, \zeta) \in \kappa \times \kappa$.
      Define $h(\xi,\zeta) = g(f(\xi))(\zeta)$ for $(\xi,\zeta) \in \kappa
      \times \kappa$.

      Let us show that $h$ is surjective onto $\cardsucc{\kappa}$.

        Every element of $\cardsucc{\kappa}$ is an element of $\image{h}{\kappa \times \kappa}$. \\
        Proof.
          Let $n$ be an element of $\cardsucc{\kappa}$.
          Take an element $\xi$ of $\kappa$ such that $n \less f(\xi)$.
          Take an element $\zeta$ of $\kappa$ such that $g(f(\xi))(\zeta) = n$.
          Indeed $g(f(\xi))$ is a surjective map from $\kappa$ onto $f(\xi)$.
          Then $n = h(\xi,\zeta)$.
        End.

        Every element of $\image{h}{\kappa \times \kappa}$ is an element of
        $\cardsucc{\kappa}$. \\
        Proof.
          Let $n$ be an element of $\image{h}{\kappa \times \kappa}$.
          We can take elements $a, b$ of $\kappa$ such that $n = h(a,b)$.
          Then $n = g(f(a))(b)$.
          $f(a)$ is an element of $\cardsucc{\kappa}$.
          Every element of $f(a)$ is an element of $\cardsucc{\kappa}$.

          Case $f(a)$ has an element.
            Then $g(f(a))$ is a surjective map from $\kappa$ onto $f(a)$.
            Hence $n \in f(a) \in \cardsucc{\kappa}$.
            Thus $n \in \cardsucc{\kappa}$.
          End.

          Case $f(a)$ has no element.
            Then $g(f(a)) = \constzero{\kappa}$.
            Hence $n$ is the empty set.
            Thus $n \in \cardsucc{\kappa}$.
          End.
        End.

        Hence $\range{h} = \image{h}{\kappa \times \kappa} = \cardsucc{\kappa}$.
      End.

      Therefore $\card{\cardsucc{\kappa}} \leq \card{\kappa \times \kappa}$
      (by \printref{SET_THEORY_06_192336220913664}).
      Indeed $\kappa \times \kappa$ and $\cardsucc{\kappa}$ are nonempty sets and $h$
      is a surjective map from $\kappa \times \kappa$ to $\cardsucc{\kappa}$.
      Consequently $\cardsucc{\kappa} \leq \kappa$.
      Contradiction.
    \end{proof}
  \end{forthel}

  \printbibliography
\end{document}
