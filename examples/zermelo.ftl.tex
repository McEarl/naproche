\documentclass{article}

\usepackage[utf8]{inputenc}
\usepackage[english]{babel}

\usepackage{meta-inf/lib/naproche-logo}
\usepackage{meta-inf/lib/naproche}
\usepackage{biblatex}
\addmhbibresource{references}

\libusepackage{naproche}

\newcommand{\pow}{\mathcal{P}}
\usepackage{biblatex}
\addmhbibresource{references}

\libusepackage{naproche}

\newcommand{\pow}{\mathcal{P}}

\usepackage{biblatex}
\usepackage{csquotes}
\addbibresource{meta-inf/lib/references.bib}

\title{Zermelo's Well-ordering Theorem}
\author{\Naproche formalization: \vspace{0.5em} \\
Marcel Schütz (2022)}
\date{}

\begin{document}
  \maketitle

  \noindent This is a formalization of \textit{Zermelo's Well-ordering Theorem},
  i.e. of the assertion that under the assumption of the axiom of choice every
  set is equinumerous to some ordinal number, where an ordinal number is
  regarded as a transitive set whose elements are transitive sets as well.
  The proof of this theorem presented here is oriented on \cite{Koepke2018}.

  On mid-range hardware \Naproche needs approximately 4 Minutes to verify this
  formalization plus approximately 15 minutes to verify the library files it
  depends on.

  \begin{forthel}
    %[prove off][check off]

    [readtex \path{foundations/sections/13_equinumerosity.ftl.tex}]

    [readtex \path{set-theory/sections/04_recursion.ftl.tex}]

    %[prove on][check on]
  \end{forthel}

  \begin{forthel}
    \begin{definition*}
      Let $X$ be a system of nonempty sets.
      A choice function for $X$ is a map $g$ such that $\dom(g) = X$ and
      $g(x) \in x$ for any $x \in X$.
    \end{definition*}

    \begin{axiom*}[Choice]
      Let $X$ be a system of nonempty sets.
      Then there exists a choice function for $X$.
    \end{axiom*}
  \end{forthel}

  \noindent In the following, for any class $A$, we write $\unboundedEnum{A}$ to
  denote the collection of all maps $f : \alpha \to A$ for some ordinal
  $\alpha$.
  Moreover, for any map $G : \unboundedEnum{A} \to A$ we say that a map $F : \Ord \to
  A$, where $\Ord$ denotes the class of all ordinals, is recursive regarding $G$
  if $F(\alpha) = G(F \restriction \alpha)$ for all $\alpha \in \Ord$.

  \begin{forthel}
    \begin{theorem*}[Zermelo]
      Every set is equinumerous to some ordinal.
    \end{theorem*}
    \begin{proof}
      Let $x$ be a set.
      Consider a choice function $g$ for $\pow{x} \setminus \set{\emptyset}$.
      For any $F \in \unboundedEnum{x \cup \set{x}}$ if
      $x \setminus \range{F} \neq \emptyset$ then $x \setminus \range{F} \in
      \dom(g)$.
      Indeed $x \setminus \range{F}$ is a subset of $x$ for any $F \in
      \unboundedEnum{x \cup \set{x}}$.
      Define \[ G(F) =
        \begin{cases}
          g(x \setminus \range{F})
          & : x \setminus \range{F} \neq \emptyset
          \\
          x
          & : x \setminus \range{F} = \emptyset
        \end{cases} \]
      for $F \in \unboundedEnum{x \cup \set{x}}$.
      Then for any $F \in \unboundedEnum{x \cup \set{x}}$ if
      $x \setminus \range{F} \neq \emptyset$ then
      $G(F) \in x \setminus \range{F}$.
      $G$ is a map from $\unboundedEnum{x \cup \set{x}}$ to $x \cup \set{x}$.
      Indeed we can show that for any $F \in \unboundedEnum{x \cup \set{x}}$ we
      have $G(F) \in x \cup \set{x}$.
        Let $F \in \unboundedEnum{x \cup \set{x}}$.
        If $x \setminus \range{F} \neq \emptyset$ then
        $G(F) \in x \setminus \range{F}$.
        If $x \setminus \range{F} = \emptyset$ then $G(F) = x$.
        Hence $G(F) \in x \cup \set{x}$.
      End.
      Hence we can take a map $F$ from $\Ord$ to $x \cup \set{x}$ that is
      recursive regarding $G$.
      For any ordinal $\alpha$ we have $F \restriction \alpha \in
      \unboundedEnum{x \cup \set{x}}$.

      For any $\alpha \in \Ord$ we have
      \[ x \setminus \image{F}(\alpha) \neq \emptyset \implies
      F(\alpha) \in x \setminus \image{F}(\alpha) \]
      and
      \[ x \setminus \image{F}(\alpha) = \emptyset \implies F(\alpha) = x. \]
      Proof.
        Let $\alpha \in \Ord$.
        We have $\image{F}(\alpha) = \{ F(\beta) \mid \beta \in \alpha \}$.
        Hence $\image{F}(\alpha) = \{ G(F \restriction \beta) \mid \beta \in \alpha \}$.
        We have $\range{F \restriction \alpha} =
        \{ F(\beta) \mid \beta \in \alpha \}$.
        Thus $\range{F \restriction \alpha} = \image{F}(\alpha)$.

        Case $x \setminus \image{F}(\alpha) \neq \emptyset$.
          Then $x \setminus \range{F \restriction \alpha} \neq \emptyset$.
          Hence $F(\alpha)
            = G(F \restriction \alpha)
            \in x \setminus \range{F \restriction \alpha}
            = x \setminus \image{F}(\alpha)$.
        End.

        Case $x \setminus \image{F}(\alpha) \neq \emptyset$.
          Then $x \setminus \range{F \restriction \alpha} = \emptyset$.
          Hence $F(\alpha)
            = G(F \restriction \alpha)
            = x$.
        End.
      Qed.

      (1) For any ordinals $\alpha, \beta$ such that $\alpha \less \beta$ and
      $F(\beta) \neq x$ we have $F(\alpha), F(\beta) \in x$ and $F(\alpha) \neq
      F(\beta)$. \\
      Proof.
        Let $\alpha, \beta \in \Ord$.
        Assume $\alpha \less \beta$ and $F(\beta) \neq x$.
        Then $x \setminus \image{F}(\beta) \neq \emptyset$.
        (a) Hence $F(\beta) \in x \setminus \image{F}(\beta)$.
        We have $\image{F}(\alpha) \subseteq \image{F}(\beta)$.
        Thus $x \setminus \image{F}(\alpha) \neq \emptyset$.
        (b) Therefore $F(\alpha) \in x \setminus \image{F}(\alpha)$.
        Consequently $F(\alpha), F(\beta) \in x$ (by a, b).
        We have $F(\alpha) \in \image{F}(\beta)$ and $F(\beta) \notin \image{F}(\beta)$.
        Thus $F(\alpha) \neq F(\beta)$.
      Qed.

      (2) There exists an ordinal $\alpha$ such that $F(\alpha) = x$. \\
      Proof.
        Assume the contrary.
        Then $F$ is a map from $\Ord$ to $x$.

        Let us show that $F$ is injective.
          Let $\alpha, \beta \in \Ord$.
          Assume $\alpha \neq \beta$.
          Then $\alpha \less \beta$ or $\beta \less \alpha$.
          Hence $F(\alpha) \neq F(\beta)$ (by 1).
          Indeed $F(\alpha), F(\beta) \neq x$.
        End.

        Thus $F$ is an injective map from some proper class to some set.
        Contradiction.
      Qed.

      Define $\Phi = \{ \alpha \in \Ord \mid F(\alpha) = x \}$.
      $\Phi$ is nonempty.
      Hence we can take a least element $\alpha$ of $\Phi$ regarding ${\in}$.
      Take $f = F \restriction \alpha$.
      Then $f$ is a map from $\alpha$ to $x$.
      Indeed for no $\beta \in \alpha$ we have $F(\beta) = x$.
      Indeed for all $\beta \in \alpha$ we have $(\beta, \alpha) \in {\in}$.

      (3) $f$ is surjective onto $x$. \\
      Proof.
        $x \setminus \image{F}(\alpha) = \emptyset$.
        Hence $\range{f}
          = \image{f}(\alpha)
          = \image{F}(\alpha)
          = x$.
      Qed.

      (4) $f$ is injective. \\
      Proof.
        Let $\beta, \gamma \in \alpha$.
        Assume $\beta \neq \gamma$.
        We have $f(\beta), f(\gamma) \neq x$.
        Hence $f(\beta) \neq f(\gamma)$ (by 1).
        Indeed $\beta \less \gamma$ or $\gamma \less \beta$.
      Qed.

      Therefore $f$ is a bijection between $\alpha$ and $x$.
      Consequently $x$ and $\alpha$ are equinumerous.
    \end{proof}
  \end{forthel}

  \printbibliography
\end{document}
