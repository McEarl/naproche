\documentclass[11pt]{article}
\usepackage{../lib/tex/naproche}

\author{Peter Koepke, University of Bonn}

\title{Ontological Axioms for \Naproche}

\begin{document}

\maketitle

\section{Basic Ideas}

The natural proof assistant \Naproche is intended to
approximate and support ordinary mathematical practices.
\Naproche employs the controlled natural language ForTheL
as its input language. Some ForTheL notions are already
built into \Naproche, as well as some basic properties of
these notions.
There are
mathematical objects, and sets and classes of mathematical
objects. Sets are classes which are objects themselves and
can thus be used in further mathematical constructions. Functions
and maps map objects to objects, where functions are those
maps which are objects.

Modelling mathematical notions by objects corresponds
to the intuition that numbers, points, etc. should
not have internal set-theoretical
structurings, contrary to purely set-theoretical
foundations of mathematics. This is also advantageous
for automated proving since it prevents proof searches to
dig into non-informative internal structurings.

As in common mathematics, sets and functions are required to
satisfy separation and replacement properties known from the
set theories of Kelley-Morse or Zermelo-Fraenkel.

The ontology presented here is more hierarchical than
ad-hoc first-order axiomatizations in some previous
\Naproche formalizations. Sometimes
this results in more complex and longer ontological checking tasks.
Controlling the complexitiy of automated proofs will be a major
issue for the further development.


\section{Importing Some Mathematical Language}

We import singular/plural forms of words that will be used in
our formalizations (\path{examples/vocabulary.ftl.tex}).
In the long run this should be replaced by
employing a proper English vocabulary. We also
import some alternative formulations for
useful mathematical phrases (\path{examples/macros.ftl.tex}).

\begin{forthel}
  [readtex \path{preliminaries/source/axioms.ftl.tex}]

  [readtex \path{preliminaries/source/vocabulary.ftl.tex}]

  [readtex \path{preliminaries/source/macros.ftl.tex}]
\end{forthel}


\section{Sets and Classes}

The notions of \textit{classes} and \textit{sets} are already
provided by \Naproche. Classes are usually defined by
abstraction terms $\{\dots\mid\dots\}$. Since such terms need to be
processed by the parser we cannot introduce them simply by first-order
definitions but have to implement them in the software.
That sets are classes which are objects, or extensionality can be proved
from inbuilt assumptions, but it is important to reprove or restate such
facts so that they can directly be accessed by the ATP.

\begin{forthel}
  Let $S, T$ denote classes.

  \begin{definition}
    The empty set is the set that has no elements.
  \end{definition}

  \begin{definition}
    A subclass of $S$ is a class $T$ such that every $x \in T$ belongs to $S$.
  \end{definition}

  Let $T \subseteq S$ stand for $T$ is a subclass of $S$.

  Let a proper subclass of $S$ stand for a subclass $T$ of $S$ such that $T \neq
  S$.

  \begin{lemma}[Class Extensionality]
    If $S$ is a subclass of $T$ and $T$ is a subclass of $S$ then $S = T$.
  \end{lemma}

  \begin{definition}
    A subset of $S$ is a set $X$ such that $X \subseteq S$.
  \end{definition}

  Let a proper subset of $S$ stand for a subset $X$ of $S$ such that $X \neq S$.

  \begin{axiom}[Separation Axiom]
    Assume that $X$ is a set.
    Let $T$ be a subclass of $X$.
    Then $T$ is a set.
  \end{axiom}

  \begin{definition}
    The intersection of $S$ and $T$ is $\class{x \in S | x \in T}$.
  \end{definition}

  Let $S \cap T$ stand for the intersection of $S$ and $T$.

  \begin{definition}
    The union of $S$ and $T$ is $\class{x | x \in S \vee x \in T}$.
  \end{definition}

  Let $S \cup T$ stand for the union of $S$ and $T$.

  \begin{definition}
    The set difference of $S$ and $T$ is $\class{x \in S | x \notin T}$.
  \end{definition}

  Let $S \setminus T$ stand for the set difference of $S$ and $T$.

  \begin{definition}
    $S$ is disjoint from $T$ iff there is no element of $S$ that is an element
    of $T$.
  \end{definition}

  \begin{definition}
    A family is a set $F$ such that every element of $F$ is a set.
  \end{definition}

  \begin{definition}
    A disjoint family is a family $F$ such that $X$ is disjoint from $Y$ for all
    nonequal elements $X, Y$ of $F$.
  \end{definition}
\end{forthel}


\section{Pairs and Products}

Since we prefer objects over sets if possible, we do not work
with Kuratowski-style set-theoretical ordered pairs, but
axiomatize them as objects.

\begin{forthel}
  \begin{axiom}
    For any objects $a, b, c, d$ if $(a,b) = (c,d)$ then $a = c$ and $b = d$.
  \end{axiom}

  \begin{definition}
    $S \times T = \class{(x,y) | \text{$x \in S$ and $y \in T$}}$.
  \end{definition}

  \begin{axiom}
    Let $X, Y$ be sets.
    Then $X \times Y$ is a set.
  \end{axiom}

  \begin{lemma}
    Let $x, y$ be objects.
    If $(x,y)$ is an element of $S \times T$ then $x$ is an element of $S$ and
    $y$ is an element of $T$.
  \end{lemma}
\end{forthel}


\section{Functions and Maps}

The treatment of functions and maps is similar to that
of sets and classes.

\begin{forthel}
  Let $f$ stand for a map.

  \begin{axiom}
    Assume that $\dom(f)$ is a set and $f(x)$ is an object for any
    $x \in \dom(f)$.
    Then $f$ is a function.
  \end{axiom}

  \begin{definition}
    Assume $S$ is a subclass of the domain of $f$.
    $f[S] = \class{f(x) | x \in S}$.
  \end{definition}

  Let the image of $f$ stand for $f[\dom(f)]$.

  \begin{definition}
    $f$ maps elements of $S$ to elements of $T$ iff $\dom(f) = S$ and $f[S]
    \subseteq T$.
  \end{definition}

  Let $f : S \rightarrow T$ stand for $f$ maps elements of $S$ to elements of
  $T$.

  \begin{axiom}[Replacement Axiom]
    Let $X$ be a set.
    Assume that $X$ is a subset of the domain of $f$.
    Then $f[X]$ is a set.
  \end{axiom}

  Let $g$ retracts $f$ stand for $g(f(x)) = x$ for all elements $x$ of
  $\dom(f)$.
  Let $h$ sections $f$ stand for $f(h(y)) = y$ for all elements $y$ of
  $\dom(h)$.

  \begin{definition}
    $f : S \leftrightarrow T$ iff $f : S \rightarrow T$ and there exists a map
    $g$ such that $g : T \rightarrow S$ and $g$ retracts $f$ and $g$ sections
    $f$.
  \end{definition}
\end{forthel}

\end{document}
